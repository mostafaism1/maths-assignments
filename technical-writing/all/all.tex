\documentclass{article}
\usepackage{amsmath, amsthm, amsfonts}
\usepackage{centernot}
\usepackage[top=2cm]{geometry}

\author{Mostafa Hassanein}
\title{
  GEN-600 Technical Writing \\
  % Assignment (1)
  }
\date{27 February 2025}
\begin{document}

\maketitle
\newpage

\section*{1.1}

Theoretical Computer Science studies the foundations of computation through formal models of computation and their realization by programming languages. In practice, choosing a suitable model of computation and a corresponding language for a particular domain can greatly streamline the programming process by increasing expressiveness, disallowing certain programming errors from occuring and providing stronger correctness guarantees. Looking ahead, I aim to contribute to the design of new and practical computational models and languages that better support emerging paradigms such as concurrent, distributed, blockchain, and high performance computing.


\newpage

\section*{2.1}

\begin{center}
  \textbf{\underline{Paragraph Analysis:}}
\end{center}

\textbf{\underline{Paragraph 1:}}
\newline

Writing is not simply speech written down on paper. Learning to write is not just a natural extension of learning to speak a language. We learned to speak our first language at home without systematic instruction; whereas, most of us had to be taught in school how to write that same language. Many adult native speakers of a language find writing difficult. A speaker speaks to a listener who is right there, nodding or frowning, or interrupting or questioning. For the writer, the reader's response is either delayed or nonexistent.
\newline

\begin{table}[ht]
  \centering
  \begin{tabular}{p{0.7\linewidth} | p{0.3\linewidth}}
    \textbf{Sentence} & \textbf{Analysis} \\
\hline
1. Writing is not simply speech written down on paper.
& Simple sentence. \\
\hline
2. Learning to write is not just a natural extension of learning to speak a language.
& Simple sentence \\
\hline
3. We learned to speak our first language at home without systematic instruction; whereas, most of us had to be taught in school how to write that same language.

& - Compound sentence with 2 main clauses.
\newline

- Main clause 1: "We learned to speak our first language at home without systematic instruction".
\newline

- Main clause 2: "most of us had to be taught in school how to write that same language". \\
\hline

4. Many adult native speakers of a language find writing difficult.
& Simple sentence. \\
\hline

5. A speaker speaks to a listener who is right there, nodding or frowning, or interrupting or questioning.
& - Compound sentence.
\newline

- Main clause: "A speaker speaks to a listener".
\newline

- Relative clause: "who is right there". \\
\hline

6. For the writer, the reader's response is either delayed or nonexistent.
& - Simple sentence.
\newline
- Phrase: "For the writer". \\
\hline
\end{tabular}
\caption{Grammatical Analysis of Paragraph 1}
\end{table}

\newpage

\textbf{\underline{Paragraph 2:}}

One of English-teaching objectives is showing how language affects text features and arrangement. The ability to visualize the features of a text , and how those features are arranged is vital to understanding when reading. When readers are able to visualize in this way, they understand and control texts at a much deeper level. This control comes through visualizing a text type and understanding how writers construct meaning through arrangement of text features. It is this understanding that is fundamental to reading comprehension. Crucially when a reader has this understanding, language is perceived not as a set of rules but as a set of options used for constructing a variety of meanings.

\begin{table}[ht]
  \centering
  \begin{tabular}{p{0.7\linewidth} | p{0.3\linewidth}}
    \textbf{Sentence} & \textbf{Analysis} \\
\hline
1. One of English-teaching objectives is showing how language affects text features and arrangement. 
& - Simple sentence. \\

\hline
2. The ability to visualize the features of a text, and how those features are arranged is vital to understanding when reading.
& - Compound sentence.
\newline

- Main clause: "The ability to visualize the features of a text is vital to understanding when reading".
\newline

- Relative clause: "and how those features are arranged". \\
\hline
3. When readers are able to visualize in this way, they understand and control texts at a much deeper level.
& - Compound sentence.
\newline

- Main clause: "they understand and control texts at a much deeper level".
\newline

- Subordinate clause: "When readers are able to visualize in this way". \\
\hline
4. This control comes through visualizing a text type and understanding how writers construct meaning through arrangement of text features.
& - Compound sentence with 2 main clauses.
\newline

- Main clause 1: "This control comes through visualizing a text type".
\newline

- Main clause 2: "understanding how writers construct meaning through arrangement of text features".
\newline

- Coordinating conjunction: "and". \\
\hline
5. It is this understanding that is fundamental to reading comprehension.
& - Compound sentence.
\newline

- Main clause: "It is this understanding".
\newline

- Relative clause: "that is fundamental to reading comprehension". \\
\hline
6. Crucially when a reader has this understanding, language is perceived not as a set of rules but as a set of options used for constructing a variety of meanings.
& - Compound sentence.

- Main clause: "language is perceived not as a set of rules but as a set of options used for constructing a variety of meanings".

- Subordinate clause: "when a reader has this understanding". \\
\hline
  \end{tabular}
  \caption{Grammatical Analysis of Paragraph 2}
\end{table}


\newpage

\section*{2.2}

\begin{center}
  \textbf{\underline{Action Verbs}}
\end{center}

1. enhance

2. boost

3. improve

4. validate

5. verify

6. predict

7. produce

8. generate

9. compile

10. map

11. filter

12. induce

13. eliminate

14. collect

15. reduce

16. strengthen

17. advance

18. support

\newpage

\section*{2.3}

\textbf{\underline{Thesis Statement:}}

Advanced type systems enhance programming languages by improving expressiveness and providing stronger correctness and safety guarantees. 
\newline

\noindent
\textbf{\underline{Topic:}} Advanced type systems
\newline

\noindent
\textbf{\underline{Comment:}} enhance programming languages by improving expressiveness and providing stronger correctness and safety guarantees
\newline

\noindent
\textbf{\underline{Keywords:}} Advanced type systems, expressiveness, correctness, safety

\newpage


% \newpage

% \section*{2#B}

% 1. Thesis Statement
% I design domain-specific programming languages that use formal semantics and advanced type systems to improve the expressiveness, safety, and problem-solving capabilities of software in specialized computational domains.

% 2. Title
% Designing Safer and More Expressive Domain-Specific Languages through Formal Semantics and Type Systems

% 3. Outline
% Topic Outline (TO)
% A topic outline lists the main points as short phrases. It’s useful for clarity and high-level structure.

% Organization Method 1: Chronological (development over time)

% Background on programming language theory

% Evolution of domain-specific languages (DSLs)

% Role of formal semantics in language design

% Importance of type systems for safety and correctness

% Case studies of existing DSLs

% My approach to designing a new DSL

% Evaluation and comparison with existing languages

% Future directions in DSL development

% Organization Method 2: Problem–Solution

% Limitations of general-purpose languages in specialized domains

% Challenges in safety and expressiveness

% Formal semantics as a foundation for precision

% Type systems as tools for correctness guarantees

% Designing DSLs to address domain-specific needs

% Implementation methodology and tooling

% Real-world application and evaluation

% Conclusion and outlook

% Sentence Outline (SO)
% A sentence outline expands each point into full sentences. It’s ideal for detailed planning before writing.

% Organization Method 1: Chronological

% Programming languages have evolved from general-purpose tools to highly specialized languages tailored for specific tasks.

% Domain-specific languages (DSLs) have gained prominence for offering focused abstractions.

% Formal semantics provide a rigorous foundation for defining language behavior.

% Type systems enhance reliability by preventing entire classes of errors.

% Several DSLs demonstrate the practical benefits of these principles.

% This research proposes a new DSL that integrates formal methods from the ground up.

% The proposed DSL is evaluated against existing tools in terms of safety, expressiveness, and usability.

% The thesis concludes with a discussion on how these techniques can guide future language development.

% Organization Method 2: Problem–Solution

% General-purpose languages often lack the expressiveness needed for complex domain-specific problems.

% Developers face challenges in maintaining safety and correctness without specialized tools.

% Formal semantics can eliminate ambiguities and guide correct implementation.

% Advanced type systems offer compile-time guarantees, reducing runtime errors.

% Domain-specific languages can provide tailored abstractions that align better with domain logic.

% This research presents a methodology for building such DSLs with formal foundations.

% A prototype implementation demonstrates improved safety and expressiveness.

% Future work includes extending the DSL to more domains and improving tooling support.

\newpage

\section*{3}

\textbf{\underline{Paper:}} "Theory and Implementation of a Functional Programming Language" by Ari Lamstein, published in the Rose-Hulman Undergraduate Mathematics Journal in 2000.
\newline

\textbf{\underline{1. Thesis Statement:}}

The theory and implementation of functional programming languages illustrate how mathematical concepts such as type theory can shape the structure and semantics of programming languages, resulting in languages that promotes expressiveness, clarity, safety, and correctness.
\newline

\textbf{\underline{2. Outline (Based on the Abstract):}}
\newline

1. Introduction to Lambda-Calculus (The mathematical framework for functional programming languages).
\newline

2. Presentation of the PPL language (A strongly typed, call-by-name functional language with support for recursion and polymorphism).
\newline

3. Compiler design for PPL.
\newline

4. Presenting the type inference algorithm
\newline

\textbf{\underline{3. Plan-Do-Check-Improve (PDCI) Analysis:}}
\newline

\quad \textbf{\underline{1. Plan:}}

\qquad Identify the requirements for designing a practical programming language that incorporates features from the theoretical study of programming language semantics.
\newline

\quad \textbf{\underline{2. Do:}}

\qquad Design the language and implement a corresponding compiler that translates.
\newline

\quad \textbf{\underline{3. Check:}}

\qquad Evaluate the language's features and the compiler's performance to ensure they align with the practical goals.
\newline

\quad \textbf{\underline{4. Improve:}}
Refine the language design and compiler implementation based on the evaluation to enhance performance.
\newline

\textbf{\underline{4. Summary:}}

This research explores how ideas from mathematics can be used to design and build a new programming language. The main focus is on functional programming, a style of coding that treats computation as the evaluation of mathematical functions.
\newline

To achieve this, the author studied the theory behind programming languages: lambda-calculus. Using this knowledge, he designed a functional language that includes key features from lambda-calculus such as higher-order functions, recursion, lexical scoping, and type inference. Finally, he outlined the necessary steps for a practical implementation of the language by providing programs for both the parser and the interpreter.
\newline

In summary, this research demonstrates how theoretical concepts can be used to develop a practical programming language. By applying mathematical principles, the author successfully created a functional programming language that is both efficient and expressive.


\newpage

\section*{4.1}
\begin{center}
  \textbf{\underline{Sentence Rewriting}}
\end{center}

1. He dropped out needed studies on account of the fact that it was very necessary for him to help support requirements of his immediate business.
\newline

\textbf{\underline{Rewrite:}} Immediate requirements of his business forced him to drop necessary studies.

\textbf{\underline{Topic:}} Immediate requirements of his business

\textbf{\underline{Method Applied:}} Rearrangement (interchange subject and object), conciseness (use compound noun)

\textbf{\underline{KWs:}} studies, business

\textbf{\underline{N/NP:}} Immediate requirements of his business
\newline
\newline

2. The subjects that are considered the most important by students are those that have been shown to be useful to them after graduation.
\newline

\textbf{\underline{Rewrite:}} The most important subjects are those that are useful to students after graduation.

\textbf{\underline{Topic:}} The most important subjects

\textbf{\underline{Method Applied:}} Conciseness (use compound noun), delete redundant words

\textbf{\underline{KWs:}} studies, business

\textbf{\underline{N/NP:}} The most important subjects
\newline
\newline

3. In the not too distant future, college freshmen must all become aware of the fact that there is a need for them to make contact with an academic adviser concerning the matter of a major study.
\newline

% \textbf{\underline{Rewrite:}} College freshmen will be required make contact with an academic adviser concerning the matter of a major study.

\textbf{\underline{Rewrite:}} Contacting an academic adviser will be required by college freshmen concerning their major study.

\textbf{\underline{Topic:}} Contacting an academic adviser

\textbf{\underline{Method Applied:}} Rearrangement, conciseness, substitution

\textbf{\underline{KWs:}} academic adviser, college, freshmen, study

\textbf{\underline{N/NP:}} Contacting an academic adviser
\newline
\newline

4. In our company there are wide-open opportunities for professional growth with a company that enjoys an enviable record for stability in the dynamic atmosphere of aerospace technology.
\newline

\textbf{\underline{Rewrite:}} Our company, which enjoys an enviable record for stability in the dynamic atmosphere of aerospace, offers wide-open opportunities for professional growth.

\textbf{\underline{Topic:}} Our company

\textbf{\underline{Method Applied:}} Rearrangement

\textbf{\underline{KWs:}} company, stability, aerospace, opportunities, technology

\textbf{\underline{N/NP:}} Our company
\newline
\newline

5. The syndicate holds the oldest and largest open market for traditional folkloric work, however, few people know how respected and valuable the artistry is.
\newline

\textbf{\underline{Rewrite:}} The respected and valuable artistry at the oldest open market for traditional folkloric work held by the syndicate is overlooked by most people!

\textbf{\underline{Topic:}} The respected and valuable artistry at the oldest open market for traditional folkloric work held by the syndicate

\textbf{\underline{Method Applied:}} Rearrangement

\textbf{\underline{KWs:}} syndicate, open market, traditional, folkloric, artistry

\textbf{\underline{N/NP:}} The respected and valuable artistry present at the oldest open market for traditional folkloric work held by the syndicate
\newline
\newline

% unfinished
6. Celebrating the local traditional folklore every summer, samples of local work are exposed in the open market. However, few people know how respected this artistry is.
\newline

\textbf{\underline{Rewrite:}} 

\textbf{\underline{Topic:}} 

\textbf{\underline{Method Applied:}} 

\textbf{\underline{KWs:}} 

\textbf{\underline{N/NP:}} 
\newline
\newline

7. The five main types of bad boss are the workaholic, the kind of person described as terrorizing, a person who communicates badly, the jellyfish type, and who insists on perfection.
\newline

\textbf{\underline{Rewrite:}} The five main types of bad bosses are the workaholic, the terrorizing, the bad communicator, the jellyfish, the perfectionist.

\textbf{\underline{Topic:}} The five main types of bad bosses

\textbf{\underline{Method Applied:}} Parallelism

\textbf{\underline{KWs:}} boss, workaholic, terrorizing, communicates, jellyfish, perfection

\textbf{\underline{N/NP:}} The five main types of bad bosses
\newline
\newline

8. Three days ago you asked us to investigate the problem of discomfort among your office workers. We have made our study. Too low humidity is apparently the main cause of your problem. Your building is steam-heated.
\newline

\textbf{\underline{Rewrite:}} In response to your request to investigate three days ago, our study concluded that the high humidity caued by steam heating is the primary cause of discomfort among your office workers.

\textbf{\underline{Topic:}} our study

\textbf{\underline{Method Applied:}} cohesion (connect related sentences into one)

\textbf{\underline{KWs:}} investiagte, discomfort, office, worker, study, humidity, problem, steam-heated

\textbf{\underline{N/NP:}} our study
\newline
\newline

\newpage

\section*{4.2}
\begin{center}
  \textbf{\underline{Correct the Verb}}
\end{center}

1. The trainees or the foreman (is, are) ….

Ans. is
\newline

2. The trainees and the foreman (is, are) …

Ans. are
\newline

3. (You and me, you and I) are invited to dinner.

Ans. You and I
\newline

(because it's in the subject position)

4. There (is, are, was, were) the chairman and the general
manager.

Ans. are
\newline

5. The healthiest specimen among the monkeys (was, were)
chosen for medical experiments.

Ans. was
\newline

6. The police (seeked, sought, seek, seeks) to find the
instigator of what happened in vain .

Ans. sought
\newline

7. I expect him to arrive early and bringing his tools with him.

Ans. I expect him to arrive early and bring his tools with him.
\newline

8. If you were (I, me, him, he), you would have made the same
mistake.

Ans. me/him
\newline

9. The train was arrived at 9:00 am.

Ans. The train arrived at 9:00 am.

\newpage

\section*{5.1}

\begin{center}
  \textbf{\underline{Paragraph Rewriting}}
\end{center}


1. Like most reorganizations, this one was not easy. It will be no surprise to many of you that this was a stressful period; both for those who participated on the reorganization team and those who waited to learn the outcome.
\newline

\textbf{\underline{Correction}}: This reorganization was not easy as expected. It was a stressful period for both those who participated on the reorganization team and those who waited to learn the outcome—a fact that will likely not surprise many of you.

\textbf{\underline{Error}}: Dangling modifier.

\textbf{\underline{Resolution}}: Changed the sentence structure to eliminate the dangling modifier.
\newline
\newline

2. Unlike readers of the academic world, most readers of the “real world” read selectively: rather than digesting a piece of writing, they skim-read most of it, skipping from one main idea to another until they come to something that particularly interests them.
\newline

\textbf{\underline{Correction}}: Unlike academic readers, most real-world readers read selectively: rather than digesting a piece of writing, they skim from one main idea to another until they find something of particular interest.

\textbf{\underline{Error}}: Lack of clarity.

\textbf{\underline{Resolution}}: Simplified structure to improve clarity.
\newline
\newline


3. When faced with two conflicting obligations, both of which appear to be justified, one approach is to try to find a way to satisfy both of them.
\newline

\textbf{\underline{Correction}}: One way to resolve two conflicting yet necessary obligations is to find a way to satisfy both of them.

\textbf{\underline{Error}}: Danglin modifier. Lack of clarity and cohesion.

\textbf{\underline{Resolution}}: Changed the sentence structure to eliminate the dangling modifier. Increase cohesion using 'yet' as a connective.
\newline
\newline

4. To study the level of the English language and the style of abstract writing, a group of persons interested in the usage of English language has reviewed the majority of the technical abstracts submitted to EGPC 14th Conference
\newline

\textbf{\underline{Correction}}: To study English proficiency and abstract writing style, a group interested in English usage has reviewed the majority of technical abstracts submitted to the EGPC 14th Conference.

\textbf{\underline{Error}}: Lack of clarity.

\textbf{\underline{Resolution}}: Used compound nouns to improve clarity.
\newline
\newline

5. After verifying the primary impression,the group decided to make further indepth analyses on nearly 50\% of the abstracts, randomly selected from the initially reviewed ones.
\newline

\textbf{\underline{Correction}}: After the group verified the primary impression, they decided to make further in-depth analyses on nearly 50\% of randomly selected abstracts from the ones initially reviewed.

\textbf{\underline{Error}}: Danglin modifier. Lack of clarity and cohesion.

\textbf{\underline{Resolution}}: Changed the sentence structure to eliminate the dangling modifier. Used compound nouns to improve clarity
\newline
\newline

\newpage

\section*{5.2}

\begin{center}
  \textbf{\underline{Correct Faulty Expressions}}
\end{center}

1. In operating the press, the hand was injured.

\textbf{\underline{Correction}}: While I was operating the press, my hand was injured.
% \textbf{\underline{Fix}}: Added a clear subject to eliminate the dangling modifier.
\newline

2. Not able to find the way, one came to my rescue.

\textbf{\underline{Correction}}: Since I was not able to find the way, someone came to my rescue.
% Fix: Clarified who was unable to find the way and added logical structure.
\newline

3. Given the circumstances, lucky we are.

\textbf{\underline{Correction}}: Given the circumstances, we are lucky.
% Fix: Reordered sentence to clearly connect modifier to subject.
\newline

4. After reading the minutes, a heated discussion started.

\textbf{\underline{Correction}}: After we read the minutes, a heated discussion started.
% Fix: Added subject to resolve dangling modifier.
\newline

5. To determine its value, the watch will be examined.

\textbf{\underline{Correction}}: To determine the watch's value, we will examine it.
% Fix: Added correct subject and revised verb form.
\newline

6. The paint is to be stored when finished.

\textbf{\underline{Correction}}: The paint is to be stored when it is finished.
% Fix: Clarified what "finished" refers to.
\newline

7. While asleep, fire started in the room.

\textbf{\underline{Correction}}: While we were asleep, a fire started in the room.
% Fix: Added subject to match the modifier “while asleep.”
\newline

8. After sitting calm, the building began to shake, and we rushed out to open areas.

\textbf{\underline{Correction}}: After we sat calmly, the building began to shake, and we rushed out to open areas.
% Fix: Clarified subject and adjusted adverb use.
\newline

9. Arriving late, the crew had already left.

\textbf{\underline{Correction}}: Because I arrived late, the crew had already left.
% Fix: Resolved dangling modifier by clarifying who arrived late.
\newline

10. Ready to pitch the camp, orders changed to move to a new location.

\textbf{\underline{Correction}}: As we were ready to pitch the camp, we received orders to move to a new location.
% Fix: Added a subject and logical connection between ideas.


\newpage

\section*{5.3}

\begin{center}
  \textbf{\underline{Write a Paragraph expanding the Outline}}
\end{center}


% \newpage

% \section*{5.}

% \textbf{\underline{Rewrite each of the following correcting style errors. Define the error type and method of redeem.}}


% 1.Like most reorganizations, this one was not easy. It will be no surprise to many of you that this was a stressful period; both for those who participated on the reorganization team and those who waited to learn the outcome.

\end{document}