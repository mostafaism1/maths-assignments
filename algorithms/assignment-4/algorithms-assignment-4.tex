\documentclass{article}
\usepackage{amsmath, amsthm, amsfonts}
\usepackage{centernot}
\usepackage{listings}
\usepackage{color}

\newtheorem{thm}{Theorem}
\newtheorem{lem}[thm]{Lemma}
\newtheorem{cor}[thm]{Corollary}
\newtheorem{rem}[thm]{Remark}
\newtheorem{remark}[thm]{Remark}
\newtheorem{conj}[thm]{Conjecture}

\definecolor{dkgreen}{rgb}{0,0.6,0}
\definecolor{gray}{rgb}{0.5,0.5,0.5}
\definecolor{mauve}{rgb}{0.58,0,0.82}

\lstset{frame=tb,
  language=Java,
  aboveskip=3mm,
  belowskip=3mm,
  showstringspaces=false,
  columns=flexible,
  basicstyle={\small\ttfamily},
  numbers=none,
  numberstyle=\tiny\color{gray},
  keywordstyle=\color{blue},
  commentstyle=\color{dkgreen},
  stringstyle=\color{mauve},
  breaklines=true,
  breakatwhitespace=true,
  tabsize=3
}

\author{Mostafa Hassanein}
\title{
  MTH-681 Analysis and Design of Algorithms \\
  Assignment (4): Divide and Conquer \\
  The Master Theorem}
\date{31 May 2025}
\begin{document}

\maketitle
\newpage

\section*{4-9}

Consider the following reucrrence:
\begin{align*}
 T(n) &= T(n/2) + 5^{\lfloor \log_5{n} \rfloor} &&\\
 T(1) &= \Theta(1)
\end{align*}

Can you solve it using the master method? If "yes", solve it; if "no", explain why. 

\begin{center}
  \textbf{\underline{Solution:}}  
\end{center}
\begin{align*}
  a &= 1 \\
  b &= 2 \\
  \log_b{a} &= \log_2{1} = 0.
\end{align*}

To simplify the analysis, assume $n$ is an integer power of 5: $n = 5^k$.
\begin{align*}
  \implies f(n) 
  &= 5^{\lfloor \log_5{n} \rfloor} &&\\
  &= 5^{\log_5{n}} &&\\
  &= n.
\end{align*}

Comparing $f(n)$ with $n^{\log_b{a}}$:
\begin{align*}
  f(n) 
  &= n &&\\
  &= \Omega(n^{0 + \epsilon}) \quad \text{for } \epsilon=0.5 &&\\
  &= \Omega(n^{\log_b{a} + \epsilon}) \quad \text{for } \epsilon=0.5.
\end{align*}

This matches the first condition for the third case of the master method.
\newline

Next, we check the regularity condition:

\begin{align*}
  f(n/b)
  &= f(n/2) &&\\
  &= n/2 &&\\
  &\leq c n \quad \text{for } c=0.6 &&\\
  &= c f(n) \quad \text{for } c=0.6.
\end{align*}

Therefore the recurrence does match the third case of the master method.
\newline

Finally, we apply the master method to find a tight bound on $T(n)$:
\begin{align*}
  T(n) = \Theta(f(n)) = \Theta(n).
\end{align*}

\end{document}