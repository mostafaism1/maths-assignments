\documentclass{article}
\usepackage{amsmath, amsthm, amsfonts}
\usepackage{centernot}
\usepackage{listings}
\usepackage{color}

\newtheorem{thm}{Theorem}
\newtheorem{lem}[thm]{Lemma}
\newtheorem{cor}[thm]{Corollary}
\newtheorem{rem}[thm]{Remark}
\newtheorem{remark}[thm]{Remark}
\newtheorem{conj}[thm]{Conjecture}

\definecolor{dkgreen}{rgb}{0,0.6,0}
\definecolor{gray}{rgb}{0.5,0.5,0.5}
\definecolor{mauve}{rgb}{0.58,0,0.82}

\lstset{frame=tb,
  language=Java,
  aboveskip=3mm,
  belowskip=3mm,
  showstringspaces=false,
  columns=flexible,
  basicstyle={\small\ttfamily},
  numbers=none,
  numberstyle=\tiny\color{gray},
  keywordstyle=\color{blue},
  commentstyle=\color{dkgreen},
  stringstyle=\color{mauve},
  breaklines=true,
  breakatwhitespace=true,
  tabsize=3
}

\author{Mostafa Hassanein}
\title{
  MTH-681 Analysis and Design of Algorithms \\
  MIDTERM EXAM}
\date{8 May 2025}
\begin{document}

\maketitle
\newpage



% \section*{1}

% \subsection*{a.}



% \begin{center}
%   \textbf{\underline{Solution:}}  
% \end{center}

% \subsection*{a.}

% \begin{lstlisting}
%   int
% \end{lstlisting}

% \newpage


% \begin{thm}
%   placeholder
% \end{thm}

% \begin{proof}(By Substitution)
%   $ $
%   \newline
  
%   PLACEHOLDER
%   \begin{align*}
%     PLACEHOLDER
%   \end{align*}
% \end{proof}

% \newpage

% \subsection*{b.}

% \begin{lem}
%     PLACEHOLDER
% \end{lem}

% \begin{proof}
%   $ $
%   \newline

%   PLACEHOLDER
% \end{proof}

% \newpage

% \begin{thm}
%   The algorithm is correct.
% \end{thm}

% \begin{proof}(By Strong Induction)
%   $ $
%   \newline

%   \textbf{\underline{Base case ($n=3$):}}
  
%   PLACEHOLDER

%   \textbf{\underline{Inductive step ($n > 3$):}}

%   PLACEHOLDER
  
% \end{proof}


% % Snippets
% % \begin{align}
% %   & X=1  \nonumber &&\\
% %   \implies& Y=2 \label{1-8.2}
% % \end{align}
% % \eqref{1-8.1} and \eqref{1-8.2} $\implies S(n) = \Theta(2^n)$.


\section*{Question 1}
\begin{thm}
  $W_t(n) = T(n) = \Theta(n \lg n)$.
\end{thm}

\begin{proof}
  $ $
  \newline

  In the worst case of this algorithm the variable "$found$" is never set to true (because on each iteration the search item is not found in the sequence S) and thus the while loop executes $n-1 = \Theta(n)$ times.
  \newline

  On each iteration we call BSearchHelp on an input of size $i$.
  \newline

  Since BSearchHelp is an $\Theta(\lg n)$ algorithm, then on each iteration we incur $\Theta(\lg i)$.
  \newline

  The total time is then given by the summation:

  \begin{align*}
    T(n) 
    &= \sum_{i=2}^{n} \Theta(\lg i) \\
    &= \sum_{i=2}^{n} c_1 \lg i \\
    &= c_1 \sum_{i=2}^{n} \lg i \\
  \end{align*}

  \textbf{\underline{Upper Bound:}}

  \begin{align*}
    T(n) 
    &= c_1 \sum_{i=2}^{n} \lg i \\
    &\leq c_1 \sum_{i=2}^{n} \lg n \\
    &\leq c_1 n \lg n
  \end{align*}

  Take $c_1 = 1$ and $n_0 = 1$, then:

  $\implies T(n) = O(n\ln n)$.
  \newline

  \textbf{\underline{Lower Bound:}}

  % Since $i \lg i$ is a non-decreasing function, then we can bound the sum using integration bounds:

  % \begin{align*}
  %   \leq \sum_{i=2}^{n} \Theta(i \lg i) \leq
  % \end{align*}

  
\end{proof}


\newpage

\section*{Question 2}

\begin{align*}
  a &= 2 \\
  b &= 8 \\
  \log_b{a} &= \log_8{2} = 3 \\
  n^{\log_b{a}} &= n^{3}
\end{align*}

1. 
\begin{align*}
  &f(n) = n \\
  \implies & f(n) = O(n^2)  \\
  \implies & f(n) = O(n^{\log_b{a} - \epsilon}) \quad \text{for } \epsilon = 1 \\
  \implies& \text{Case 1}
\end{align*}
\newline

2. 
\begin{align*}
  &f(n) = n^3 \\
  \implies & f(n) = \Theta(n^3)  \\
  \implies & f(n) = \Theta(n^{\log_b{a}}) \\
  \implies& \text{Case 2}
\end{align*}
\newline

3. 
\begin{align}
  &f(n) = n^4 \nonumber \\
  \implies & f(n) = \Omega(n^3) \nonumber \\
  \implies & f(n) = \Omega(n^{\log_b{a} + \epsilon}) \quad \text{for } \epsilon = 1 \label{2.1.1}
\end{align}

and:

\begin{align}
  af(n/b) 
  &= 2f(n/8) \nonumber \\
  &= \frac{2}{8^3} n^3 \nonumber \\
  &< c f(n) = c n^3 \quad \text{for } c = \frac{3}{8^3} \label{2.1.2}
\end{align}

$\eqref{2.1.1}$ and $\eqref{2.1.2}$ $\implies \text{Case 3}$.
\newline


4. 
\begin{align*}
  &f(n) = \frac{1}{\log{n}} * n^3 \\
  \implies & \left(f(n) \neq \Theta(n^3)\right) \quad 
  \land \quad \left(\forall \epsilon: f(n) \neq \Omega(n^{3 + \epsilon})\right) \quad 
  \land \quad \left(\forall \epsilon: f(n) \neq O(n^3 - \epsilon)\right) \\
  \implies& \text{The Master theorem does not apply.}
\end{align*}
\newline

5. 
\begin{align*}
  &f(n) = \log{n} * n^3 \\
  \implies & \left(f(n) \neq \Theta(n^3)\right) \quad 
  \land \quad \left(\forall \epsilon: f(n) \neq \Omega(n^{3 + \epsilon})\right) \quad 
  \land \quad \left(\forall \epsilon: f(n) \neq O(n^3 - \epsilon)\right) \\
  \implies& \text{The Master theorem does not apply.}
\end{align*}


\newpage

\section*{Question 3}

\textbf{\underline{Algorithm}:}

1. We iterate over the elements of the set A, and we build up the partial sums into an array S.

2. For each possible pair of indices $i,j$ where $1 \leq i < j \leq n$, compute the difference $s[j] - s[i]$ and check if it equals $k$.




\end{document}