\documentclass{article}
\usepackage{amsmath, amsthm, amsfonts}
\usepackage{centernot}
\author{Mostafa Hassanein}
\title{
  MTH-681 Algorithms \\
  Assignment (1): Asymptotic Analysis}
\date{8 March 2025}
\begin{document}

\maketitle
\newpage

\section*{1-8}

Using two different methods, prove that:
\begin{align*}
  \sum_{i=1}^{i=n} 2^i = \Theta(2^n)
\end{align*}

\begin{proof}(First proof: By explicitly finding constants $c$ and $n_0$)
  $ $

  The series $\sum_{i=1}^{i=n} 2^i$ is a geometric series which has a closed form solution given by:
  \begin{align*}
    S(n) = \sum_{i=1}^{i=n} 2^i = \frac{2^n - 2}{2-1} = 2^n - 2.
  \end{align*}

  \textbf{\underline{Upperbound:}}

  Take $c=1$, $n_0=1$, then $\forall n \in Z^+$ and $n> n_0$:
  \begin{align}
    & S(n) = 2^n - 2\leq 2^n \nonumber &&\\
    \implies& S(n) = O(2^n). \label{1-8.1}
  \end{align}


  \textbf{\underline{Lowerbound:}}

  Take $c=1/2$, $n_0=2$, then $\forall n \in Z^+$ and $n> n_0$:
  \begin{align}
    & S(n) = 2^n - 2 \geq 2^{n-1}  \nonumber &&\\
    \implies& S(n) = \Omega(2^n). \label{1-8.2}
  \end{align}

  \textbf{\underline{Tightbound:}}

  \eqref{1-8.1} and \eqref{1-8.2} $\implies S(n) = \Theta(2^n)$.
\end{proof}

\newpage

\begin{proof}(Second proof: Using integrals)
  $ $

  In this proof, we again start from the closed-form solution:
  \begin{align*}
    S(n) = \sum_{i=1}^{i=n} 2^i = \frac{2^n - 2}{2-1} = 2^n - 2.
  \end{align*}

  but use integration to get a lower and upper bounds.
  \newline

  Since $S(n)$ is monotically increasing, then the bounds are given by:
  \newline

  \begin{align*}
    % 1
    &\int_{x=0}^{x=n} 2^x dx 
    &&\leq S(n)
    &&\leq \int_{x=1}^{x=n+1} 2^x dx &&\\
    % 2
    \implies& \frac{1}{\ln(2)} \left[2^n - 1\right]
    &&\leq S(n)
    &&\leq \frac{1}{\ln(2)} \left[2^{n+1} - 2\right] &&\\
    % 3
    \implies& S(n) = \Theta(2^n).
  \end{align*}
  
\end{proof}

\end{document}