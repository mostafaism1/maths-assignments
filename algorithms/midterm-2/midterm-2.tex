\documentclass{article}
\usepackage{amsmath, amsthm, amsfonts}
\usepackage{centernot}
\usepackage{listings}
\usepackage{color}

\newtheorem{thm}{Theorem}
\newtheorem{lem}[thm]{Lemma}
\newtheorem{cor}[thm]{Corollary}
\newtheorem{rem}[thm]{Remark}
\newtheorem{remark}[thm]{Remark}
\newtheorem{conj}[thm]{Conjecture}

\definecolor{dkgreen}{rgb}{0,0.6,0}
\definecolor{gray}{rgb}{0.5,0.5,0.5}
\definecolor{mauve}{rgb}{0.58,0,0.82}

\lstset{frame=tb,
  language=Java,
  aboveskip=3mm,
  belowskip=3mm,
  showstringspaces=false,
  columns=flexible,
  basicstyle={\small\ttfamily},
  numbers=none,
  numberstyle=\tiny\color{gray},
  keywordstyle=\color{blue},
  commentstyle=\color{dkgreen},
  stringstyle=\color{mauve},
  breaklines=true,
  breakatwhitespace=true,
  tabsize=3
}

\author{Mostafa Hassanein}
\title{
  MTH-681 Analysis and Design of Algorithms \\
  SECOND MIDTERM EXAM}
\date{22 JUNE 2025}
\begin{document}

\maketitle
\newpage



% \section*{1}

% \subsection*{a.}



% \begin{center}
%   \textbf{\underline{Solution:}}  
% \end{center}

% \subsection*{a.}

% \begin{lstlisting}
%   int
% \end{lstlisting}

% \newpage


% \begin{thm}
%   placeholder
% \end{thm}

% \begin{proof}(By Substitution)
%   $ $
%   \newline
  
%   PLACEHOLDER
%   \begin{align*}
%     PLACEHOLDER
%   \end{align*}
% \end{proof}

% \newpage

% \subsection*{b.}

% \begin{lem}
%     PLACEHOLDER
% \end{lem}

% \begin{proof}
%   $ $
%   \newline

%   PLACEHOLDER
% \end{proof}

% \newpage

% \begin{thm}
%   The algorithm is correct.
% \end{thm}

% \begin{proof}(By Strong Induction)
%   $ $
%   \newline

%   \textbf{\underline{Base case ($n=3$):}}
  
%   PLACEHOLDER

%   \textbf{\underline{Inductive step ($n > 3$):}}

%   PLACEHOLDER
  
% \end{proof}


% % Snippets
% % \begin{align}
% %   & X=1  \nonumber &&\\
% %   \implies& Y=2 \label{1-8.2}
% % \end{align}
% % \eqref{1-8.1} and \eqref{1-8.2} $\implies S(n) = \Theta(2^n)$.


\section*{Question 1}
\subsection*{1.}
\begin{align*}
  &2^{n+1} = 2*2^n < 3 * 2^n \\
  \implies& \exists c = 3, n_0 = 0, \text{such that } \forall n > n_0:  2^{n+1} < c*2^n \\
  \implies& 2^{n+1} = O(2^n).
\end{align*}


\subsection*{2.}
\begin{align*}
  &2^{2n} = (2^n)^2 = 2^n * 2^n \\
  \implies& \not \exists c, n_0, \text{such that } \forall n > n_0:  2^{2n} < c*2^n \\
  \implies& 2^{2n} = \Omega(2^n) \\
  \implies& 2^{2n} \neq O(2^n).
\end{align*}

\subsection*{3.}
We can obtain an upper-bound on the summation using an integral bound:
\begin{align*}
  \sum_{k=1}^{n} k \ln{k} 
  &\leq \int_{1}^{n+1} x \ln{x} dx \\
  &= \left[ \frac{1}{2} x^2 \ln{x} - \frac{1}{4} x^2 \right]_{x = 1} ^{n+1} \\
  &= \frac{1}{2} (n+1)^2 \ln{(n+1)} - \frac{1}{4} (n+1)^2 + c \\
  &\leq \frac{1}{2} (n+1)^2 \ln{(n+1)}.
\end{align*}

Next, we use the limit test:
\begin{align*}
  L 
  &= \lim_{n \to \infty} \frac{\frac{1}{2} (n+1)^2 \ln{(n+1)}}{n^2 \ln{n}} \\
  &= \lim_{n \to \infty} \frac{\frac{1}{2} (n^2 +2n + 1) \ln{(n+1)}}{n^2 \ln{n}} \\
  &= \lim_{n \to \infty} \frac{\frac{1}{2} n^2 \ln{(n+1)} + n \ln{(n+1)} + \frac{1}{2}\ln{(n+1)}}{n^2 \ln{n}} \\
  &= \lim_{n \to \infty} \frac{\frac{1}{2} n^2 \ln{(n+1)}}{n^2 \ln{n}} \\
  &= \lim_{n \to \infty} \frac{\frac{1}{2} n^2 \ln{n} + \ln{1}}{n^2 \ln{n}} \\
  &= \lim_{n \to \infty} \frac{\frac{1}{2} n^2 \ln{n}}{n^2 \ln{n}} = \frac{1}{2}.
\end{align*}
$\implies \sum_{k=1}^{n} k \ln{k} = \Theta(n^2 \ln{n})$
$\implies \sum_{k=1}^{n} k \ln{k} = O(n^2 \ln{n})$.

\newpage

\section*{Question 2}

\subsection*{1.}

Scrable[S, 1, 2]:
    
    mid = 1

    After 2 calls to merge sort help S won't change: S = [5,2]

    The first call to Scrambe(S, 1, 1), results in: S[1] = S[1] * S[2] = 5*2 = 10.

    The second and third calls to Scrable won't do anything.

    The fourth call to scrable, results in: S[1] = S[1] * S[2] = 10*2 = 20.

    The final calls to merge sort help won't do anything.

    Therefore, the final result is: S = [20, 2]


\subsection*{2.}
\begin{align*}
  T(n) 
  &= 4T(\frac{n}{2}) + 4\Theta(\frac{n}{2} \lg{\frac{n}{2}}) \\
  &= 4T(\frac{n}{2}) + \Theta(n \lg n).
\end{align*}


\subsection*{3.}
\begin{align*}
  a &= 4 \\
  b &= 2 \\
  \lg_b{a} &= \lg_2{4} = 2 \\
  n^{\lg_b{a}} &= n^2 \\
  f(n) &= n \lg{n}
\end{align*}

\begin{align*}
  \implies& f(n) = O(n^{2 -\epsilon}), \text{ for } \epsilon = 0.5 \\
  \implies& \text{First Case} \\
  \implies& T(n) = \Theta(n^2).
\end{align*}


\newpage

\section*{Question 3}

\textbf{\underline{Algorithm}:}

\begin{lstlisting}
  Sequence LCS(s1, s2) {
  return LCSRecursive(s1, 1, m, s2, 1, n, 0)
}


Sequence LCSRecursive(Sequence s1, Index low_1, Index high_1, Sequence s2, Index low_2, Index high_2, Sequence partialResult) {
  if (low_1 > high_1 || low_2 > high_2) {
    return partialResult;
  }
  else {
    if (s[low_1] == s[low_2]) {
      LCSRecursive(s1, low_1+1, high_1, s2, low_2+1, high_2, partialResult + 1);
    }
    else if (s[low_1] < s[low_2]) {
      LCSRecursive(s1, low_1+1, high_1, s2, low_2, high_2, partialResult);
    }
    else {
      LCSRecursive(s1, low_1, high_1, s2, low_2+1, high_2, partialResult);
    }
  }
}
\end{lstlisting}

\end{document}