\documentclass{article}
\usepackage{amsmath}
\usepackage{amsfonts}
\author{Mostafa Hassanein}
\title{Linear Algebra Assignments}
\date{1st April 2020}
\begin{document}

\maketitle

\newpage

\section*{1.2.13}

No. Because the inverse is undefined for some elements in $V$.
\newline

\noindent 
To show this, let's begin by finding the identity element. Let $a=(a_1,a_2) \in V$, we want to find the element $id=(id_1, id_2)$ such that:
\begin{align}
  a+id&=a \nonumber \\
  (a_1, a_2) + (id_1, id_2) &= (a_1, a_2) \nonumber \\
  (a_1 + id_1, a_2id_2) &= (a1, a_2) \label{1.2.13.identity}
\end{align}
\eqref{1.2.13.identity} implies that $id=(0,1)$.
\newline

\noindent
Next, let's find the inverse element $a^{-1}$ for any $a \in V$:
\begin{align}
  a + a^{-1} &= id \nonumber \\
  (a_1, a_2) + (a^{-1}_1, a^{-1}_2) &= (0, 1) \nonumber \\
  (a_1 + a^{-1}_1, a_2*a^{-1}_2) &= (0, 1) \label{1.2.13.inverse}
\end{align}
\eqref{1.2.13.inverse} implies that $a^{-1}_1 = -a_1$ and $a^{-1}_2=\frac{1}{a_2}$. But whenever $a_2=0$, $a^{-1}_2$ is undefined.
\newline

\noindent
Since the inverse is not defined for all $a \in V$, then $V$ is not a vector space.

\section*{1.2.14}
Yes.
\newline

\noindent
Since the set of elements in $V$ over $\mathbb{R}$ is a subset of $V$ over $\mathbb{C}$, and $V$ over $\mathbb{C}$ is a vector space, this problem can be reduced to checking whether $V$ over $\mathbb{R}$ is a subspace of $V$ over $\mathbb{C}$. We only need to check 2 conditions: closure under vector addition and closure and scalar multiplication.
\newline

\noindent

\subsection*{Closure under Vector Addition}
Let $a$ and $b$ $\in V$, then:
\begin{equation*}
  a + b = (a_1, a_2) + (b_1, b_2)= (a_1 + b_1, a_2 + b_2)  
\end{equation*}
\noindent
Since $a$ and $b \in \mathbb{R} \Rightarrow$ $a_1$, $a_2$, $b_1$, and $b_2$ $\in \mathbb{R} \Rightarrow$ $a_1+b_1$ and $a_2+b_2$ $\in \mathbb{R} \Rightarrow$ $a+b \in V$. Therefore V is closed under vector addition.

\subsection*{Closure under Scalar Multiplication}
Let $a \in V$ and $c \in \mathbb{R}$, then:
\begin{equation*}
  ca = c(a_1, a_2) = (c*a_1, c* a_2)  
\end{equation*}
\noindent
Since $a \in V \Rightarrow$ $a_1$, $a_2$ $\in \mathbb{R} \Rightarrow$ $ca_1$ and $ca_2$ $\in \mathbb{R} \Rightarrow$ $ca \in V$. Therefore V is closed under scalar multiplication.

\section*{1.2.19}
No. Because it fails axiom 8: $(k_1 + k_2)u = k_1u + k_2u$.

\subsection*{Counter Example}
Let $u=(1,1)$, $k_1=1$, and $k_2=1$:

\begin{align}
  (k_1 + k_2)u &= (1+1)(1,1) \nonumber \\
  &= (2)(1,1) \nonumber \\
  &= (2, \frac{1}{2}) \label{1.2.19.lhs}
\end{align}

\noindent
But,
\begin{align}
  k_1u + k_2u &= 1(1,1) + 1(1,1) \nonumber \\
  &= (1,1) + (1,1) \nonumber \\
  &= (2, 2) \label{1.2.19.rhs}
\end{align}

\noindent
Since \eqref{1.2.19.lhs} $\neq$ \eqref{1.2.19.rhs}, $V$ is not a vector space.
\end{document}