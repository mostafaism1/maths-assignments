\documentclass{article}
\usepackage{amsmath, amsthm, amsfonts}
\usepackage{centernot}
\author{Mostafa Hassanein}
\title{
  MTH-684 Logic \\
  Assignment (1): Propositional Logic}
\date{25 September 2024}
\begin{document}

\maketitle
\newpage

\section*{1-3}

\begin{proof}(By Strong Induction)

  We will show that this inequality holds using proof by strong induction on the number of connectives $n$ in a WFF $\phi$:
  \newline

  \underline{Base case $(n = 0)$:}
  \newline

  This case represents \textbf{atomic expressions}.

  From the definition of $sub$ on \textbf{atomic expressions}, it is clear that $sub(\phi) = \{A\}$, where $A$ is an atomic expression.

  \noindent
  $\implies$
  $|sub(\phi)| = 1$

  \noindent
  $\implies$
  $|sub(\phi)| \leq 2n + 1 = 1$ holds.
  \newline

  \underline{Inductive step ($n = k+1$):}
  \newline

  This case represents \textbf{composite expressions}.

  Suppose the induction hypothesis holds for $n \leq k$, we will do a case analysis on all connective types to show that it also holds for $n = k+1$.
  \newline

  \noindent
  \textbf{\underline{(i): $\lnot$}}
  \begin{align*}
    &\phi = \lnot \psi \\
    \implies& sub(\phi) = sub(\psi) \cup \{ \lnot \psi\} \\
    \implies& |sub(\phi)| \leq |sub(\psi)| + 1 \\
    \implies& |sub(\phi)| \leq (2k + 1) + 1 = 2(k+1) \leq 2(k+1) + 1.
  \end{align*}


  \noindent
  \textbf{\underline{(ii): $\land$}} 
  \begin{align*}
    &\phi = \psi \land \omega \\
    \implies& sub(\phi) = sub(\psi) \cup sub(\omega) \cup \{ (\psi \land \omega) \} \\
    \implies& |sub(\phi)| \leq |sub(\psi)| + |sub(\omega)| + 1 \\
    \implies& |sub(\phi)| \leq [2r + 1] + [2(k-r) + 1] + 1 \\
    \implies& |sub(\phi)| \leq 2k + 3 = 2(k+1) + 1.
  \end{align*}


  \noindent
  \textbf{\underline{(iii - v): $\lor$, $\implies$, $\iff$}} 
  \newline
  
  These cases all have similar proof to case $ii$.
  \newline

  \noindent
  This closes the induction, and thus $|sub(\phi)| \leq 2n + 1$ is true.

\end{proof}

\end{document}