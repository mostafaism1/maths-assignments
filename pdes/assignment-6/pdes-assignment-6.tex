\documentclass{article}
\usepackage{amsmath, amsthm, amsfonts}
\usepackage{centernot}
\usepackage{derivative}
\author{Mostafa Hassanein}
\title{
  MTH-632 PDEs \\
  Assignment (6): \\ Chapter 9: Numerical Analysis of PDEs \\
  Finite Difference (FD) Technique}
\date{20 Dec 2024}
\begin{document}

\maketitle
\newpage

\section*{1.}

The governing algebraic (finite difference) equation for the steady state temperature is given by:
\begin{align*}
  T_{i,j} = \frac{T_{i-1,j} + T_{i+1,j} + T_{i,j-1} + T_{i,j+1}}{4}
\end{align*}

We start with an initial guess for the temperature at the interior nodes to be zero.
\newline

\underline{iteration 1:}
\begin{align*}
  &T_{1,1} = \frac{T_{0,1} + T_{2,1} + T_{1,0} + T_{1,2}}{4}
  = \frac{75 + 0 + 0 + 0}{4} = 18.75 &&\\
  &T_{2,1} = \frac{T_{1,1} + T_{3,1} +  T_{2,0} + T_{2,2}}{4}
  = \frac{0 + 50 + 0 + 0}{4} = 12.5 &&\\
  &T_{1,2} = \frac{T_{0,2} + T_{2,2} +  T_{1,1} + T_{1,3}}{4}
  = \frac{75 + 0 + 0 + 100}{4} = 43.75 &&\\
  &T_{2,2} = \frac{T_{1,2} + T_{3,2} +  T_{2,1} + T_{2,3}}{4}
  = \frac{0 + 50 + 0 + 100}{4} = 37.5 &&\\
  &\left|\epsilon_{a_{11}}\right| = 100\%.
\end{align*}

\underline{iteration 2:}
\begin{align*}
  &T_{1,1} = \frac{T_{0,1} + T_{2,1} + T_{1,0} + T_{1,2}}{4}
  = \frac{75 + 12.5 + 0 + 43.75}{4} = 32.81 &&\\
  &T_{2,1} = \frac{T_{1,1} + T_{3,1} +  T_{2,0} + T_{2,2}}{4}
  = \frac{18.75 + 50 + 0 + 37.5}{4} = 26.56 &&\\
  &T_{1,2} = \frac{T_{0,2} + T_{2,2} +  T_{1,1} + T_{1,3}}{4}
  = \frac{75 + 37.5 + 18.75 + 100}{4} = 57.81 &&\\
  &T_{2,2} = \frac{T_{1,2} + T_{3,2} +  T_{2,1} + T_{2,3}}{4}
  = \frac{43.75 + 50 + 12.5 + 100}{4} = 51.56 &&\\
  &\left|\epsilon_{a_{11}}\right| = \left|\frac{32.81 - 18.75}{32.81}\right| = 43\%.
\end{align*}

\underline{iteration 3:}
\begin{align*}
  &T_{1,1} = \frac{T_{0,1} + T_{2,1} + T_{1,0} + T_{1,2}}{4}
  = \frac{75 + 26.56 + 0 + 57.81}{4} = 39.84 &&\\
  &T_{2,1} = \frac{T_{1,1} + T_{3,1} +  T_{2,0} + T_{2,2}}{4}
  = \frac{32.81 + 50 + 0 + 51.56}{4} = 33.59 &&\\
  &T_{1,2} = \frac{T_{0,2} + T_{2,2} +  T_{1,1} + T_{1,3}}{4}
  = \frac{75 + 51.56 + 32.81 + 100}{4} = 64.84 &&\\
  &T_{2,2} = \frac{T_{1,2} + T_{3,2} +  T_{2,1} + T_{2,3}}{4}
  = \frac{57.81 + 50 + 26.56 + 100}{4} = 58.59 &&\\
  &\left|\epsilon_{a_{11}}\right| = \left|\frac{39.84 - 32.81}{39.84}\right| = 17.6\%.
\end{align*}

\underline{iteration 4:}
\begin{align*}
  &T_{1,1} = \frac{T_{0,1} + T_{2,1} + T_{1,0} + T_{1,2}}{4}
  = \frac{75 + 33.59 + 0 + 64.84}{4} = 43.36 &&\\
  &T_{2,1} = \frac{T_{1,1} + T_{3,1} +  T_{2,0} + T_{2,2}}{4}
  = \frac{39.84 + 50 + 0 + 58.59}{4} = 37.11 &&\\
  &T_{1,2} = \frac{T_{0,2} + T_{2,2} +  T_{1,1} + T_{1,3}}{4}
  = \frac{75 + 58.59 + 39.84 + 100}{4} = 68.36 &&\\
  &T_{2,2} = \frac{T_{1,2} + T_{3,2} +  T_{2,1} + T_{2,3}}{4}
  = \frac{64.84 + 50 + 33.59 + 100}{4} = 62.11 &&\\
  &\left|\epsilon_{a_{11}}\right| = \left|\frac{43.36 - 39.84}{43.36}\right| = 8.1\%.
\end{align*}

\underline{iteration 5:}
\begin{align*}
  &T_{1,1} = \frac{T_{0,1} + T_{2,1} + T_{1,0} + T_{1,2}}{4}
  = \frac{75 + 37.11 + 0 + 68.36}{4} = 45.12 &&\\
  &T_{2,1} = \frac{T_{1,1} + T_{3,1} +  T_{2,0} + T_{2,2}}{4}
  = \frac{43.36 + 50 + 0 + 62.11}{4} = 38.87 &&\\
  &T_{1,2} = \frac{T_{0,2} + T_{2,2} +  T_{1,1} + T_{1,3}}{4}
  = \frac{75 + 62.11 + 43.36 + 100}{4} = 70.12 &&\\
  &T_{2,2} = \frac{T_{1,2} + T_{3,2} +  T_{2,1} + T_{2,3}}{4}
  = \frac{68.36 + 50 + 37.11 + 100}{4} = 63.87 &&\\
  &\left|\epsilon_{a_{11}}\right| = \left|\frac{45.12 - 43.36}{45.12}\right| = 3.9\%.
\end{align*}

\underline{iteration 6:}
\begin{align*}
  &T_{1,1} = \frac{T_{0,1} + T_{2,1} + T_{1,0} + T_{1,2}}{4}
  = \frac{75 + 38.87 + 0 + 70.12}{4} = 45.98 &&\\
  &T_{2,1} = \frac{T_{1,1} + T_{3,1} +  T_{2,0} + T_{2,2}}{4}
  = \frac{45.12 + 50 + 0 + 63.87}{4} = 39.75 &&\\
  &T_{1,2} = \frac{T_{0,2} + T_{2,2} +  T_{1,1} + T_{1,3}}{4}
  = \frac{75 + 63.87 + 45.12 + 100}{4} = 71 &&\\
  &T_{2,2} = \frac{T_{1,2} + T_{3,2} +  T_{2,1} + T_{2,3}}{4}
  = \frac{70.12 + 50 + 38.87 + 100}{4} = 64.75 &&\\
  &\left|\epsilon_{a_{11}}\right| = \left|\frac{45.98 - 45.12}{45.98}\right| = 1.8\%.
\end{align*}

\underline{iteration 7:}
\begin{align*}
  &T_{1,1} = \frac{T_{0,1} + T_{2,1} + T_{1,0} + T_{1,2}}{4}
  = \frac{75 + 39.75 + 0 + 71}{4} = 46.44 &&\\
  &T_{2,1} = \frac{T_{1,1} + T_{3,1} +  T_{2,0} + T_{2,2}}{4}
  = \frac{45.98 + 50 + 0 + 64.75}{4} = 40.18 &&\\
  &T_{1,2} = \frac{T_{0,2} + T_{2,2} +  T_{1,1} + T_{1,3}}{4}
  = \frac{75 + 64.75 + 45.98 + 100}{4} = 71.43 &&\\
  &T_{2,2} = \frac{T_{1,2} + T_{3,2} +  T_{2,1} + T_{2,3}}{4}
  = \frac{71 + 50 + 39.75 + 100}{4} = 65.19 &&\\
  &\left|\epsilon_{a_{11}}\right| = \left|\frac{46.44 - 45.98}{46.44}\right| = 0.99\% &&\\
  &\left|\epsilon_{a_{21}}\right| = \left|\frac{40.18-39.75}{40.18}\right| = 1\% &&\\
  &\left|\epsilon_{a_{12}}\right| = \left|\frac{71.43-71}{71.43}\right| = 0.6\% &&\\
  &\left|\epsilon_{a_{22}}\right| = \left|\frac{65.19-64.75}{65.19}\right| = 0.6\%.
\end{align*}


\section*{2.}
$ $

We use an explicit scheme.
\newline

The finite difference equation for Poisson's equation in 3D is given by:
\begin{align*}
  T_{i,j,k}
  = 
  \frac
  {
  T_{i-1,j,k} + T_{i+1,j,k}
  +
  T_{i,j-1,k} + T_{i,j+1,k}
  +
  T_{i,j,k-1} + T_{i,j,k-1}
  - h^2 f_{i,j,k}
  }
  {6}
\end{align*}

We start with an initial guess for the temperature at the interior nodes to be zero.
\newline

\underline{iteration 1:}
\begin{align*}
  T_{1,1,1}
  &= 
  \frac
  {
  T_{0,1,1} + T_{2,1,1}
  +
  T_{1,0,1} + T_{1,2,1}
  +
  T_{1,1,0} + T_{1,1,2}
  - h^2 f_{1,1,1}
  }
  {6} &&\\
  &=
  \frac
  {
  0 + 0
  +
  0 + 0
  +
  0 + 0
  - \frac{1}{36}(-10)
  }
  {6}
  = \frac{10}{216} &&\\
  &\vdots
\end{align*}

There are 25 points interior points, the other points can be computed in a similar manner.
\newline

On each iteration we should compute the maximum error, and if the error is low enough (say less than 1\%), we stop, otherwise we continue onto the next iteration.


\section*{5.}
$ $

To solve the PDE:
\begin{align*}
  c_t = D c_{xx} - U c_x - k c  
\end{align*}

using finite difference, we compute the finite difference approximation to the derivatives as follows:
\begin{align*}
  &c_t = \frac{c_{i}^{l+1} - c_{i}^{l}}{\Delta t} &&\\
  &c_x = \frac{c_{i+1}^{l} - c_{i}^{l}}{\Delta x} &&\\
  &c_{xx} = \frac{c_{i+1}^{l} - 2c_{i}^{l} + c_{i-1}^{l}}{\Delta x} &&\\
\end{align*}

Next, we plug these approximations to the PDE:
\begin{align*}
  &\frac{c_{i}^{l+1} - c_{i}^{l}}{\Delta t}
  =
  D \frac{c_{i+1}^{l} - 2c_{i}^{l} + c_{i-1}^{l}}{\Delta x^2}
  -
  U \frac{c_{i+1}^{l} - c_{i}^{l}}{\Delta x}
  -
  k c_{i}^{l} &&\\
  \implies& c_{i}^{l} 
  [k
  -\frac{1}{\Delta t}
  + \frac{2D}{\Delta x^2}
  - \frac{U}{\Delta x}
  ]
  +
  c_{i-1}^{l}
  [- \frac{D}{\Delta x^2}]
  +
  c_{i+1}^{l}
  [-\frac{D}{\Delta x^2}
  +
  \frac{U}{\Delta x}
  ]
  +
  c_{i}^{l+1}
  [\frac{1}{\Delta t}]
  =
  0
\end{align*}

Plugging in the following values for the constants:
\begin{align*}
  &k = 0.15 &D = 100 &&U = 1 &&\\
  &\Delta x = 1 &\Delta t = 0.005
\end{align*}

we get:
\begin{align*}
  -\frac{17}{20} c_{i}^{l} 
  -
  100 c_{i-1}^{l}
  -
  99 c_{i+1}^{l}
  +
  200 c_{i}^{l+1}
  =
  0
\end{align*}

Solving for $c_{i}^{l+1}$:
\begin{align*}
  c_{i}^{l+1}
  =
  \frac{1}{2} c_{i-1}^{l}
  +
  \frac{17}{4000} c_{i}^{l} 
  +
  \frac{99}{200} c_{i+1}^{l}
\end{align*}

We can nnow use the Liebmann's (Gauss-Seidel) method to solve for the concentrations iteratively.

\underline{iteration 1, timestep 1:}
\begin{align*}
  &c_{1}^{1}
  =
  \frac{1}{2} c_{0}^{0}
  +
  \frac{17}{4000} c_{1}^{0} 
  +
  \frac{99}{200} c_{2}^{0}
  =
  -\frac{1}{2} 100
  +
  \frac{17}{4000} 0
  +
  \frac{99}{200} 0
  =
  50 &&\\
  &c_{2}^{1}
  =
  \frac{1}{2} c_{1}^{0}
  +
  \frac{17}{4000} c_{2}^{0} 
  +
  \frac{99}{200} c_{3}^{0}
  =
  \frac{1}{2} 0
  +
  \frac{17}{4000} 0
  +
  \frac{99}{200} 0
  =
  0 &&\\
  &\vdots &&\\
  &c_{9}^{1}
  =
  \frac{1}{2} c_{8}^{0}
  +
  \frac{17}{4000} c_{9}^{0} 
  +
  \frac{99}{200} c_{10}^{0}
  = 
  \frac{1}{2} 0
  +
  \frac{17}{4000} 0
  +
  \frac{99}{200} 0
  = 0
\end{align*}

\underline{iteration 1, timestep 2:}
\begin{align*}
  &c_{1}^{2}
  =
  \frac{1}{2} c_{0}^{1}
  +
  \frac{17}{4000} c_{1}^{1} 
  +
  \frac{99}{200} c_{2}^{1}
  =
  \frac{1}{2} 100
  +
  \frac{17}{4000} 50
  +
  \frac{99}{200} 0
  =
  50.21 &&\\
  &c_{2}^{2}
  =
  \frac{1}{2} c_{1}^{1}
  +
  \frac{17}{4000} c_{2}^{1} 
  +
  \frac{99}{200} c_{3}^{1}
  =
  \frac{1}{2} 50
  +
  \frac{17}{4000} 0
  +
  \frac{99}{200} 0
  =
  25 &&\\
  &\vdots
\end{align*}

This should be continued until the final timestep (timestep 9). Then, the same steps should be iterated until the maximum error falls below the required threshold.

\end{document}