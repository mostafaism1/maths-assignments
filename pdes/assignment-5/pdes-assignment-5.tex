\documentclass{article}
\usepackage{amsmath, amsthm, amsfonts}
\usepackage{centernot}
\usepackage{derivative}
\author{Mostafa Hassanein}
\title{
  MTH-632 PDEs \\
  Assignment (5): \\ Chapter 8: Separation of Variables (Non-Homogeneous)}
\date{18 Dec 2024}
\begin{document}

\maketitle
\newpage

\section*{i.}
\subsection*{2.}

\underline{Step 1: Transform the problem to eliminate BC inhomogeneities}
\newline

BCs are already homogeneous.
\newline

\noindent
\underline{Step 2: Compute the eigenfunctions of the homogeneous PDE}
\newline

From the previous chapter, we know that PDE/eigenvalue problem:
\begin{align*}
  &\phi''(x) = -\lambda \phi(x) &&\\
  &\phi(0) = 0 &&\\
  &\phi(\pi) = 0
\end{align*}

has solutions given by the following eigenfunctions:
\begin{align*}
  &\phi_n(x) = \cos(\sqrt{\lambda} x) = \cos(n x), \quad \lambda^2 = n^2, \quad n \in \{0, 1, \ldots\}.
\end{align*}

\noindent
\underline{Step 3: Expand the source function in the eigenfunction basis to obtain $s_n(t)$}
\newline

\begin{align*}
  &s(x, t) = \sum_{n=0}^{\infty} s_n(t) \phi_n(x) &&\\
  \implies&e^{-t} = \sum_{n=0}^{\infty} s_n(t) \cos(n x) &&\\
  \implies&s_n(t) = \frac{\int_{x=0}^{\pi} e^{-t} \cos(n x) dx}{\int_{x=0}^{\pi} \cos^2(n x) dx} &&\\
  \implies& s_n(t) = \begin{cases}
    e^{-t}, &n=0 \\
    0, &n \neq 0
  \end{cases}
\end{align*}

\noindent
\underline{Step 4: Expand the solution in the eigenfunction basis}
\newline

\begin{align*}
  u(x, t) &= \sum_{n=0}^{\infty} u_n(t) \phi_n(x).
\end{align*}

\noindent
\underline{Step 5: Compute $u_{t}$ and $u_{xx}$}
\newline
\begin{align*}
  &u_{t} = \sum_{n=0}^{\infty} \dot{u}_n(t) \phi_n(x) &&\\
  &u_{xx} = \sum_{n=0}^{\infty} u_n(t) \phi''_n(x) = - n^2 \sum_{n=0}^{\infty} u_n(t) \phi_n(x).
\end{align*}

\noindent
\underline{Step 6: Expand the initial condition in the eigenfunction basis to obtain $u_n(0)$}
\newline

\begin{align*}
  &u(x, 0) = \sum_{n=0}^{\infty} u_n(0) \phi_n(x) &&\\
  \implies& \cos(2x) = \sum_{n=0}^{\infty} u_n(0) \phi_n(x) &&\\
  \implies&u_n(0) = \frac{\int_{x=0}^{\pi} \cos(2x) \cos(n x) dx}{\int_{x=0}^{\pi} \cos^2(n x) dx} &&\\
  \implies& u_n(0) = \begin{cases}
    1, &n = 2 \\
    0, &n \neq 2
  \end{cases}
\end{align*}

\noindent
\underline{Step 7: Substitute back into the PDE}
\newline
\begin{align*}
  &u_t = u_{xx} + e^{-t} &&\\
  \implies& \sum_{n=0}^{\infty} \dot{u}_n(t) \phi_n(x) 
  = -\lambda^2 \sum_{n=0}^{\infty} u_n(t) \phi_n(x)
  + \sum_{n=0}^{\infty} s_n(t) \phi_n(x) &&\\
  \implies& \dot{u}_n(t) \phi_n(x) 
  = - n^2 u_n(t) \phi_n(x)
  + s_n(t) \phi_n(x) &&\\
  \implies&\begin{cases}
    \dot{u}_0(t) = s_0(t) = e^{-t}  \\
    \dot{u}_n(t) = - n^2 u_n(t), \quad n \neq 0
  \end{cases} \\
\end{align*}

\noindent
\underline{Step 8: Solve the resulting ODEs}
\newline
\begin{align*}
  &u_n(x, t) = \begin{cases}
    A - e^{-t}, &n=0 \\
    C_n e^{-n^2 t}, &n \neq 0
  \end{cases}
\end{align*}

Comparing with the initial conditions we determine the constants as:
\begin{align*} 
  &A = 1 \\
  &C_n = \begin{cases}
    1, &n=2 \\
    0, &n \neq 2
  \end{cases}
\end{align*}


\noindent
\underline{Step 8: Form the complete solution}
\newline

\begin{align*}
  u(x, t) &= \sum_{n=0}^{\infty} u_n(t) \phi_n(x) &&\\
  &= u_0(t) \phi_0(x)
  + u_2(t) \phi_2(x) &&\\
  &= 1 - e^{-t} 
  + e^{-4t} \cos(2x)
\end{align*}
\newpage




\section*{ii.}
\subsection*{2.}

\underline{Step 1: Transform the problem to eliminate BC inhomogeneities}
\newline

BCs are already homogeneous.
\newline

\noindent
\underline{Step 2: Compute the eigenfunctions of the homogeneous PDE}
\newline

From the previous chapter, we know that PDE/eigenvalue problem:
\begin{align*}
  &\phi''(x) = -\lambda \phi(x) &&\\
  &\phi(0) = 0 &&\\
  &\phi(\pi) = 0
\end{align*}

has solutions given by the following eigenfunctions:
\begin{align*}
  &\phi_n(x) = \sin(\sqrt{\lambda} x) = \sin(n x), \quad \lambda^2 = n^2, \quad n \in \{1, \ldots\}.
\end{align*}

\noindent
\underline{Step 3: Expand the source function in the eigenfunction basis to obtain $s_n(t)$}
\newline

\begin{align*}
  &s(x, t) = \sum_{n=1}^{\infty} s_n(t) \phi_n(x) &&\\
  \implies&cos(\omega t) = \sum_{n=1}^{\infty} s_n(t) \sin(n x) &&\\
  \implies&s_n(t) = \frac{\int_{x=0}^{\pi} \cos(\omega t) \sin(n x) dx}{\int_{x=0}^{\pi} \sin^2(n x) dx} &&\\
  \implies& s_n(t) = \begin{cases}
    \frac{4}{n \pi} \cos(\omega t), &n \text{ is odd} \\
    0, &n \text{ is even}
  \end{cases}
\end{align*}


\noindent
\underline{Step 4: Expand the solution in the eigenfunction basis}
\newline

\begin{align*}
  u(x, t) &= \sum_{n=1}^{\infty} u_n(t) \phi_n(x).
\end{align*}

\noindent
\underline{Step 5: Compute $u_{t}$, $u_{tt}$ and $u_{xx}$}
\newline
\begin{align*}
  &u_{t} = \sum_{n=1}^{\infty} \dot{u}_n(t) \phi_n(x) &&\\
  &u_{tt} = \sum_{n=1}^{\infty} \ddot{u}_n(t) \phi_n(x) &&\\
  &u_{xx} = \sum_{n=1}^{\infty} u_n(t) \phi''_n(x) = - n^2 \sum_{n=1}^{\infty} u_n(t) \phi_n(x).
\end{align*}

\noindent
\underline{Step 6: Expand the initial condition in the eigenfunction basis to obtain $u_n(0)$ and $\dot{u}_n(0)$}
\newline

\begin{align*}
  &u(x, 0) = \sum_{n=1}^{\infty} u_n(0) \phi_n(x) &&\\
  \implies& f(x) = \sum_{n=1}^{\infty} u_n(0) \phi_n(x) &&\\
  \implies& u_n(0) = \frac{\int_{x=0}^{\pi} f(x) \sin(nx)}{\int_{x=0}^{\pi} \sin^2(nx)} &&\\
  \implies& u_n(0) = \frac{2}{\pi} \int_{x=0}^{\pi} f(x) \sin(nx).
\end{align*}

\begin{align*}
  &\dot{u}(x, 0) = \sum_{n=1}^{\infty} \dot{u}_n(0) \phi_n(x) &&\\
  \implies&0 = \sum_{n=1}^{\infty} \dot{u}_n(0) \phi_n(x) &&\\
  \implies&\dot{u}_n(0) = \frac{\int_{x=0}^{\pi} 0 * \sin(nx)}{\int_{x=0}^{\pi} 0 * \sin^2(nx)} = 0.
\end{align*}

\noindent
\underline{Step 7: Substitute back into the PDE}
\newline
\begin{align*}
  &u_{tt} - c^2 u_{xx} + \beta u_{t} = \cos(\omega t) &&\\
  \implies&\sum_{n=1}^{\infty} \ddot{u}_n(t) \phi_n(x)
  + c^2 n^2 \sum_{n=1}^{\infty} u_n(t) \phi_n(x)
  + \beta \sum_{n=1}^{\infty} \dot{u}_n(t) \phi_n(x)
  = \sum_{n=1}^{\infty} \frac{4}{(2n - 1) \pi} \cos(\omega t) \phi_n(x) &&\\
  \implies&\begin{cases}
    \ddot{u}_n(t)
    + c^2 n^2  u_n(t)
    + \beta  \dot{u}_n(t)
    =  \frac{4}{n \pi} \cos(\omega t), &n \text{ is odd} \\
    \ddot{u}_n(t)
    + c^2 n^2 u_n(t)
    + \beta \dot{u}_n(t)
    = 0, &n \text{ is even}.
  \end{cases}
\end{align*}
\newline

\noindent
\underline{Step 8: Solve the resulting ODEs}
\newline

\underline{n is odd:}
\begin{align*}
  u_n(t) = u_n^h + u_n^p  
\end{align*}

\underline{Compue the homogeneous solution $u_n^h$:}
\newline

We guess a solution of the form $u_n^h = Ae^{\mu_{n} t}$ and substitute into the ODE:
\begin{align*}
  &\mu_{n}^2 A e^{\mu_{n} t} + \beta \mu_{n} A e^{\mu_{n} t} + c^2 n^2 A e^{\mu_n t} = 0 &&\\
  \implies&\mu_{n}^2 + \beta \mu_{n} + c^2 n^2  = 0 &&\\
  \implies&\mu_{n} = \frac{-\beta \pm \sqrt{\beta^2 - 4 c^2 n^2}}{2}
\end{align*}

\noindent
At this point, we assume the discriminant is negative, and thus we have 2 complex roots:

\begin{align*}
  \mu_{n} = \frac{-\beta \pm i \sqrt{4 c^2 n^2 - \beta^2}}{2}
\end{align*}
\begin{align*}
  u_n^h &= Re \left[A_1 e^{\frac{-\beta + i \sqrt{4 c^2 n^2 - \beta^2}}{2} t}
  +
  A_2 e^{\frac{-\beta - i \sqrt{4 c^2 n^2 - \beta^2}}{2} t}
  \right] &&\\
  &= Re \left[e^{\frac{-\beta}{2} t}
  \left(A_1 e^{\frac{i \sqrt{4 c^2 n^2 - \beta^2}}{2} t}
  +
  A_2 e^{\frac{-i \sqrt{4 c^2 n^2 - \beta^2}}{2} t} \right)
  \right] &&\\
  &= e^{\frac{-\beta}{2} t}
  \left(A_1 \cos{\frac{\sqrt{4 c^2 n^2 - \beta^2}}{2} t}
  +
  A_2 \cos{\frac{\sqrt{4 c^2 n^2 - \beta^2}}{2} t} \right) &&\\
  &= A e^{\frac{-\beta}{2} t}
  \cos({\frac{\sqrt{4 c^2 n^2 - \beta^2}}{2} t + \phi}) &&\\
\end{align*}

\underline{Compue the particular solution $u_n^p$:}
\newline

We guess a solution of the form $u_n^p = B_1 \sin(\omega t) + B_2 \cos(\omega t)$ and substitute into the ODE:

\begin{align*}
  &-\omega^2 [B_1 \sin(\omega t) +  B_2 \cos(\omega t)]
  + \beta \omega [B_1 \cos(\omega t) - B_2 \sin(\omega t)] \\
  & + c^2n^2 [B_1 \sin(\omega t) + B_2 \cos(\omega t)]
  =
  \frac{4}{n \pi} \cos(\omega t) \\
  \implies&\sin(\omega t)
  [-\omega^2 B_1 - \beta \omega B_2 + c^2 n^2 B_1] 
  +
  \cos(\omega t)
  [-\omega^2 B_2 + \beta \omega B_1 + c^2 n^2 B_2]
  = \frac{4}{n \pi} \cos(\omega t) \\
  \implies& -\omega^2 B_1 - \beta \omega B_2 + c^2 n^2 B_1 = 0 
  \quad \text{and} \quad
  -\omega^2 B_2 + \beta \omega B_1 + c^2 n^2 B_2 = \frac{4}{n \pi} \\
  \implies& B_1 [-\omega^2 + c^2 n^2] + B_2 [-\beta \omega] = 0 
  \quad \text{and} \quad
  B_1 [\beta \omega] + B_2 [-\omega^2 + c^2 n^2]
  = \frac{4}{n \pi}
\end{align*}

This is a $2x2$ system that can be solved for $B_1$ and $B_2$.
\newline

\underline{n is even:}

This case has the same solution as $u_n^h$ for the odd n case.
\newline

\noindent
\underline{Step 8: Determine $\phi$ by comparing with the initial conditions}
\newline

\noindent
\underline{Step 9: Form the complete solution}
\newline

\begin{align*}
  u(x, t) 
  &=
  \sum_{n=1}^{\infty} u_n^h(t) \phi_n(x)
  +
  \sum_{n=1}^{\infty} u_{2n-1}^p(t) \phi_{2n-1}(x) \\
  &= 
  \sum_{n=1}^{\infty} A_n e^{\frac{-\beta}{2} t}
  \cos({\frac{\sqrt{4 c^2 n^2 - \beta^2}}{2} t} + \phi)
  \sin(nx)
  +
  \sum_{n=1}^{\infty} [B_{1_{2n-1}} \sin(\omega t) + B_{2_{2n-1}} \cos(\omega t)] \sin((2n-1)x).
\end{align*}
\newpage

\end{document}