\documentclass{article}
\usepackage{amsmath}
\usepackage{amsfonts}
\usepackage{amsthm}
\author{Mostafa Hassanein}
\title{Abstract Algebra Assignment (3): Basic Concepts of Permutations}
\date{25 November 2023}
\begin{document}

\maketitle

\newpage

\section*{1.}

\begin{proof}
  We use strong induction on m.
  \newline

  \noindent
  \textbf{\underline{Base case ($m=1$):}} 
  
  $L.H.S. = (p_1)^{-1} = R.H.S.$
  \newline
  
  \noindent
  \textbf{\underline{Inductive step ($m>1$):}} 
  
  We assume P(n) is true for all $1 \leq n \leq m$ and use it to prove P(m+1).
  For $m+1$, we have:
  \begin{align*}
    L.H.S. = (p_1 \circ p_2 \circ ... \circ p_m \circ p_{m+1})^{-1} &= ((p_1 \circ p_2 \circ ... \circ p_m) \circ p_{m+1})^{-1} & \\
    &= p_{m+1}^{-1} \circ (p_1 \circ p_2 \circ ... \circ p_m)^{-1} & \\
    &= p_{m+1}^{-1} \circ (p_m^{-1} \circ p_{m-1}^{-1} \circ ... \circ p_1^{-1}) & \text{(Because P(2) is true)} \\
    &= p_{m+1}^{-1} \circ p_m^{-1} \circ p_{m-1}^{-1} \circ ... \circ p_1^{-1} & \text{(Because P(m) is true)} \\
    &= R.H.S.
  \end{align*}
  
\end{proof}

\section*{2.}

\begin{proof}
  Given a permutation $p$ of $n$ symbols, we can be express the permutation as a product of $k$ disjoint cycles. We can also express any cycle of length $l$ as a product of $l-1$ transpositions. Therefore, we can express any permutation as a product of $t$ transpositions, where:
  \begin{align*}
    t &= \sum_{i=1}^{k} (l_i - 1) \\
      &= \sum_{i=1}^{k} l_i - \sum_{i=1}^{k} 1 \\
      &= n - k \\
  \end{align*}

  \noindent
  Therefore, $p = t_1t_2...t_(n-k)$.
  
  \noindent
  Adding any pair transpositions that are inverses of each other will leave the permutation the same.
  So, $p$ can be expressed as $(n-k) + 2a$ transpositions, where $a \in Z^+$. Since adding an even number does not change the parity, then the parity depends only on the value of $n-k$ which is unique to the permutation $p$.

\end{proof}

\section*{3.}

\begin{proof}
  Since $A_n$ is a subset of $S_n$, we need only prove that $A_n$ is a subgroup of $S_n$.
  \newline

  \noindent
  i. Closure: The product 2 even permutations is even, therefore $A_n$ is closed.
  \newline

  \noindent
  ii. Associativity: Associativity is satisfied for $S_n$ so it is also satisfied for $A_n$.
  \newline

  \noindent
  iii. Identity: The identity permutation can be expressed as a product on $n$ disjoint cycles.
  
  \noindent
  $\Rightarrow$ The identity permutation can be expressed as a product of $n-n=0$ transpositions.
  
  \noindent
  $\Rightarrow$ The identity permutation is even.
  
  \noindent
  $\Rightarrow$ The identity permutation belongs to $A_n$.
  \newline

  \noindent
  iv. Inverse: For any permutation $p = p_1 \circ p_2 \circ ... \circ p_m$, we have $p^{-1} = p_m^{-1} \circ p_{m-1}^{-1} \circ ... p_1^{-1}$.

  \noindent
  $\Rightarrow$ The inverse of an even permutation is also even.

  \noindent
  $\Rightarrow \forall a \in A_n \ \exists a^{-1}: \ a \circ a^{-1} = id$.
  \newline

  \noindent
  i, ii, iii, iv $\Rightarrow$ $A_n$ is a subgroup of $S_n$.
  \newline

  \noindent
  Next, to show that $|A_n|=n!/2$, it suffices to show that the number of even permutation $|A_n|$ is equal to the number of odd permutations $|B_n|$.
  \newline

  \noindent
  We use the fact that the product of an odd permutation with an even permutation is an odd permutation to construct a bijection from $A_n$ to $B_n$ and thus conclude that they must have the same number of elements.
  \newline

  \noindent
  Let $\tau$ be any transposition in $S_n$ (we know that one exists assuming $n > 1$), then $\tau$ is an odd permutation; and let $f: A_n \rightarrow B_n$ be a function defined as $f(a) = \tau \circ a$.
  \newline

  \noindent
  \textbf{\underline{Injectivity:}}
  $f(a_1) = f(a_2) \Rightarrow \tau \circ a_1 = \tau \circ a_2$
  
  \noindent
  $\Rightarrow \tau^{-1} \circ (\tau \circ a_1) = \tau^{-1} \circ (\tau \circ a_2)$ 

  \noindent
  $\Rightarrow (\tau^{-1} \circ \tau )\circ a_1 = (\tau^{-1} \circ \tau) \circ a_2$ 
  
  \noindent
  $\Rightarrow a_1 = a_2$
  
  \noindent
  $\Rightarrow \ f$ is injective.
  \newline

  \noindent
  \textbf{\underline{Surjectivity:}}
  Let $b \in B_n$ $\Rightarrow (\tau^{-1} \circ b)$ is an even permutation and $f(\tau^{-1} \circ b) = \tau \circ (\tau^{-1} \circ b)= (\tau \circ \tau^{-1}) \circ b = b$
  
  \noindent
  $\Rightarrow \ f$ is surjective.
  \newline

  \noindent
  $f$ is injective and surjective $\Rightarrow$ $f$ is bijective.
\end{proof}

\section*{4.} 
There are $4!=24$ permutation for $S_4$:
\begin{align*}
  &\boldsymbol{P_{x}} &&\boldsymbol{|P_{x}|} &&\boldsymbol{Parity} \\
  &P_1 = \begin{pmatrix}
    1 & 2 & 3 & 4 \\
    1 & 2 & 3 & 4
  \end{pmatrix} = (1) (2) (3) (4) &&1 &&Even \\
  &P_2 = \begin{pmatrix}
    1 & 2 & 3 & 4 \\
    1 & 2 & 4 & 3
  \end{pmatrix} = (3 \ 4) &&2 &&Odd
  \\
  &P_3 = \begin{pmatrix}
    1 & 2 & 3 & 4 \\
    1 & 3 & 2 & 4
  \end{pmatrix} = (2 \ 3) &&2 &&Odd
  \\
  &P_4 = \begin{pmatrix}
    1 & 2 & 3 & 4 \\
    1 & 3 & 4 & 2
  \end{pmatrix} = (2 \ 3 \ 4) = (2 \ 3) (2 \ 4) &&3 &&Even
  \\
  &P_5 = \begin{pmatrix}
    1 & 2 & 3 & 4 \\
    1 & 4 & 2 & 3
  \end{pmatrix} = (2 \ 4 \ 3) = (2 \ 4) (2 \ 3) &&3 &&Even
  \\
  &P_6 = \begin{pmatrix}
    1 & 2 & 3 & 4 \\
    1 & 4 & 3 & 2
  \end{pmatrix} = (2 \ 4) &&2 &&Odd
  \\
  &P_7 = \begin{pmatrix}
    1 & 2 & 3 & 4 \\
    2 & 1 & 3 & 4
  \end{pmatrix} = (1 \ 2) &&2 &&Odd
  \\
  &P_8 = \begin{pmatrix}
    1 & 2 & 3 & 4 \\
    2 & 1 & 4 & 3
  \end{pmatrix} = (1 \ 2) (3 \ 4) &&2 &&Even
  \\
  &P_9 = \begin{pmatrix}
    1 & 2 & 3 & 4 \\
    2 & 3 & 1 & 4
  \end{pmatrix} = (1 \ 2 \ 3) = (1 \ 2) (1 \ 3) &&3 &&Even
  \\
  &P_{10} = \begin{pmatrix}
    1 & 2 & 3 & 4 \\
    2 & 3 & 4 & 1
  \end{pmatrix} = (1 \ 2 \ 3 \ 4) = (1 \ 2) (1 \ 3) (1 \ 4) &&4 &&Odd
  \\
  &P_{11} = \begin{pmatrix}
    1 & 2 & 3 & 4 \\
    2 & 4 & 1 & 3
  \end{pmatrix} = (1 \ 2 \ 4 \ 3) = (1 \ 2) (1 \ 4) (1 \ 3) &&4 &&Odd
  \\
  &P_{12} = \begin{pmatrix}
    1 & 2 & 3 & 4 \\
    2 & 4 & 3 & 1
  \end{pmatrix} = (1 \ 2 \ 4) = (1 \ 2) (1 \ 4) &&3 &&Even
  \\
  &P_{13} = \begin{pmatrix}
    1 & 2 & 3 & 4 \\
    3 & 1 & 2 & 4
  \end{pmatrix} = (1 \ 3 \ 2) = (1 \ 3) (1 \ 2) &&3 &&Even
  \\
  &P_{14} = \begin{pmatrix}
    1 & 2 & 3 & 4 \\
    3 & 1 & 4 & 2
  \end{pmatrix} = (1 \ 3 \ 4 \ 2) = (1 \ 3) (1 \ 4) (1 \ 2) &&4 &&Odd
  \\
  &P_{15} = \begin{pmatrix}
    1 & 2 & 3 & 4 \\
    3 & 2 & 1 & 4
  \end{pmatrix} = (1 \ 3) &&2 &&Odd
  \\
  &P_{16} = \begin{pmatrix}
    1 & 2 & 3 & 4 \\
    3 & 2 & 4 & 1
  \end{pmatrix} = (1 \ 3 \ 4) = (1 \ 3) (1 \ 4) &&3 &&Even
  \\
  &P_{17} = \begin{pmatrix}
    1 & 2 & 3 & 4 \\
    3 & 4 & 1 & 2
  \end{pmatrix} = (1 \ 3) (2 \ 4) &&2 &&Even
  \\
  &P_{18} = \begin{pmatrix}
    1 & 2 & 3 & 4 \\
    3 & 4 & 2 & 1
  \end{pmatrix} = (1 \ 3 \ 2 \ 4) = (1 \ 3) (1 \ 2) (1 \ 4) &&4 &&Odd
  \\
  &P_{19} = \begin{pmatrix}
    1 & 2 & 3 & 4 \\
    4 & 1 & 2 & 3
  \end{pmatrix} = (1 \ 4 \ 3 \ 2) = (1 \ 4) (1 \ 3) (1 \ 2) &&4 &&Odd
  \\
  &P_{20} = \begin{pmatrix}
    1 & 2 & 3 & 4 \\
    4 & 1 & 3 & 2
  \end{pmatrix} = (1 \ 4 \ 2) = (1 \ 4) (1 \ 2) &&3 &&Even
  \\
  &P_{21} = \begin{pmatrix}
    1 & 2 & 3 & 4 \\
    4 & 2 & 1 & 3
  \end{pmatrix} = (1 \ 4 \ 3) = (1 \ 4) (1 \ 3) &&3 &&Even
  \\
  &P_{22} = \begin{pmatrix}
    1 & 2 & 3 & 4 \\
    4 & 2 & 3 & 1
  \end{pmatrix} = (1 \ 4) &&2 &&Odd
  \\
  &P_{23} = \begin{pmatrix}
    1 & 2 & 3 & 4 \\
    4 & 3 & 1 & 2
  \end{pmatrix} = (1 \ 4 \ 2 \ 3) = (1 \ 4) (1 \ 2) (1 \ 3) &&4 &&Odd
  \\
  &P_{24} = \begin{pmatrix}
    1 & 2 & 3 & 4 \\
    4 & 3 & 2 & 1
  \end{pmatrix} = (1 \ 4) (2 \ 3) &&2 &&Even
\end{align*}

\noindent
The parity of a permutation is the parity of the number of transpositions required to represent the permutation.
\newline

\noindent
The order of a permutation is the smallest positive integer $m$ such that $P^m = I$.
It can be computed by first factoring the permutation as a product of disjoint cycles, then the order $m$ of the permutation becomes the L.C.M. of orders of these cycles.
\newline

\noindent
The elements of $A_4$ are those permutations of $S_4$ with an even parity, and they form a subgroup of $S_4$.
\newline

\section*{5.}
\subsection*{i.}
\begin{proof}
  A transposition is a permutation that exchanges 2 elements and leaves all the other elements unchanged. Let $\tau$ be a transposition that exchanges the 2 elements $x$ and $y$, so that $x$ becomes $y$ and $y$ becomes $x$. When applying $\tau$ twice to $x$, the first application will move $x$ into $y$ and the second application will move $y$ to $x$, so that $x$ moves to $x$. Similarly for $y$, the first application will move $y$ into $x$ and the second application will move $x$ to $y$, so that $y$ moves to $y$. This implies that $\tau\tau=I$, which imply that $\tau = \tau^{-1}$

\end{proof}

\subsection*{ii.}
\begin{proof}
  The identity permutation on $n$ symbols moves each symbol into itself.
  
  \noindent
  $\Rightarrow$ The identity permutation has $n$ disjoint cycles each of length $1$. \label{sec:my-nice-section}
  \newline

  \noindent
  We also have that any cycle of length $n$ can be expressed as a product of $n-1$ transpositions.
  
  \noindent
  $\Rightarrow$ Any cycle of length $1$ can be expressed as a product of $0$ transpositions.
  \newline

  \noindent
  $\Rightarrow$ The identity permutation can be written as a product of $n*0 = 0$ transpositions.
  
  \noindent
  $\Rightarrow$ The identity permutation is an even permutation.

\end{proof}

\section*{7.}
\subsection*{i.}
$P_1 = (1 \ 3 \ 5) (2 \ 4) (6 \ 8 \ 7)$

\noindent
$|P_1| = LCM(3, 2, 3) = 6$

\subsection*{ii.}
$P_2 = (1 \ 4 \ 2) (5 \ 6)$

\noindent
$|P_2| = LCM(3, 2) = 6$

\subsection*{iii.}
$P_3 = (1 \ 3 \ 5 \ 8) (2 \ 7) (4 \ 6 \ 9)$

\noindent
$|P_3| = LCM(4, 2, 3) = 12$

\subsection*{iv.}
$P_4 = (1 \ 3 \ 4) (5 \ 7)$

\noindent
$|P_4| = LCM(3, 2) = 6$

\section*{8.}
\subsection*{i.}
Inversions of $\langle 3, 1, 4, 2 \rangle = \{(3, 1), (3, 2), (4, 2) \}$
\newline

\noindent
Inversions of $\langle 1, 2, 4, 5, 6, 7, 8, 3 \rangle = \{(4, 3), (5, 3), (6, 3), (7, 3), (8, 3) \}$
\newline

\noindent
Inversions of $\langle 4, 5, 3, 2, 1, 8, 6, 7, 9 \rangle = \{(4, 1), (4, 2), (4, 3), (5, 1), (5, 2), (5, 3), (3, 1), (3, 2), (2, 1), (8, 6), (8, 7) \}$
\end{document}