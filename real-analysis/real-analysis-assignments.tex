\documentclass{article}
\usepackage{amsmath, amsthm, amsfonts}

\newtheorem{innercustomthm}{Theorem}
\newenvironment{customthm}[1]
  {\renewcommand\theinnercustomthm{#1}\innercustomthm}
  {\endinnercustomthm}

\author{Mostafa Hassanein}
\title{Real Analysis Assignments}
\date{9 January 2024}
\begin{document}

\maketitle

\newpage

\section*{1.1.7}
$A_n$ is the set of all positive multiples of $(n+1)$.

\subsection*{a.}

\begin{flalign*}
  A_1 \cap A_2 &= \{ x: \text{x is a positive multiple of both 2 and 3} \} &&\\
  &= \{ x: \text{x is a positive multiple of 6} \} &&\\
  &= A_5 &&\\
  &= \{ 6, 12, 18, \ldots \}.
\end{flalign*}


\subsection*{b.}
\subsubsection*{i.}

\begin{flalign*}
  \cup \{ A_n: n \in \textbf{N} \} &= \{ x: \text{x is a positive multiple of 2 or 3 or 4 or \dots} \} &&\\
  &= \textbf{N} - \{1\}. &&\\
\end{flalign*}

\subsubsection*{ii.}

\begin{flalign*}
  \cap \{ A_n: n \in \textbf{N} \} &= \{ x: \text{x is a positive multiple of 2 and 3 and 4 and \dots} \} &&\\
  &= \emptyset \text{ (because } \forall n \in N \ n \notin A_n \text{)}. &&\\
\end{flalign*}

\section*{1.1.22}
\subsection*{a.}

\begin{proof}
  $ $

  Let $x_1, x_2 \in A$, and assume $f(x_1) = f(x_2)$

  $\implies g(f(x_1)) = g(f(x_2))$

  $\implies (g\circ f)(x_1) = (g\circ f)(x_2)$
  
  $\implies x_1 = x_2$, by injectivity of $g \circ f$

  $\implies$ f is injective.

\end{proof}

\subsection*{b.}
\begin{proof}
  $ $

  Let $c \in C$

  $\implies \exists a \in A: (g\circ f)(a) = c$

  $\implies \exists a \in A: g(f(a)) = c$

  $\implies \exists b \in B: g(b) = c$

  $\implies$ g is surjective.  

\end{proof}


\section*{1.3.7}
\textbf{\underline{Proof:}}
\newline

\noindent
\textbf{\underline{Forward direction}}: If a set $T_1$ is denumerable, then there exists a bijection from $T1$ onto a denumerable set $T_2$. 
\newline

\noindent 
Since $T_1$ is denumerable, then there exists a bijection $f_1$ from $N$ onto $T_1$.

\noindent
And since $T_2$ is denumerable, then there exists a bijection $f_2$ from $N$ onto $T_2$.

\noindent
Also, since $f_1$ is a bijection, then $f_1^{-1}$ exists and is a bijection from $T_1$ onto $N$.

\noindent
Then, the function defined by $f_2 \circ f_1^{-1}$ is a bijection from $T1$ onto $T_2$, since the compositiomn of bijective functions is bijective.
\newline

\noindent
\textbf{\underline{Reverse direction}}: If there exists a bijection from a set $T1$ onto a denumerable set $T_2$, then $T_1$ is denumerable.
\newline

\noindent
Let $g$ be the bijection from $T1$ to $T2$.

\noindent
Since $T_2$ is denumerable, then there exists a bijection $f_2$ from $N$ onto $T_2$.

\noindent
Also, since $g$ is a bijection, then $g^{-1}$ exists and is a bijection from $T_2$ onto $T_1$.

\noindent
Then, the function defined as $f_1 := g^{-1} \circ f_2$ is a bijection from $N$ onto $T_1$, since the compositiomn of bijective functions is bijective.

\noindent
This implies that the set $T_1$ is denumerable.
\newline

\noindent
This completes the proof.
\qed

\section*{1.3.8}
Let's define the set $S_i$ as $S_i:=\{i\}$, then for each $i$, $S_i$ is a finite set (of cardinality = 1). 

\noindent
But the union $\cup_{i=1}^{\infty}S_i = N$ is infinite, because $N$ is infinite.
\newline

\section*{2.1.4}
By the trichotomy of $a$, we have three cases: (i) $a < 0$, (ii) $a = 0$, and (iii) $a > 0$.
\newline

\noindent
\textbf{\underline{(i):}} $a<0 \Rightarrow a \cdot a > 0 \Rightarrow a \cdot a > a \Rightarrow a \cdot a \not= a$. Therefore, $a$ cannot be less than 0.
\newline

\noindent
\textbf{\underline{(ii):}} $a=0 \Rightarrow a \cdot a = a$, because $0 \cdot 0 = 0$.
\newline

\noindent
\textbf{\underline{(iii):}} $a>0$ and $a \cdot a = a \Rightarrow a^{-1} \cdot a \cdot a = a^{-1} \cdot a \Rightarrow a = 1$.
\newline

\noindent
Therefore, $a=0$ or $a=1$.
\qed
\newline

\section*{2.1.23}
\textbf{\underline{Proof:}}
\newline

\noindent
\textbf{\underline{Forward direction:}} For $a>0$, $b>0$, and $n \in N$: If $a<b$, then $a^n<b^n$.
\newline

\noindent
We use induction.
\newline

\noindent
\textbf{\underline{Base case ($n=1$):}} This case is trivially true because it is given by the hypothesis: $a<b \iff a^1 < b^1$.
\newline

\noindent
\textbf{\underline{Inductive step ($n>1$):}} By the induction hypothesis, we have: 
\begin{equation*}
  a<b \Rightarrow a^n < b^n
\end{equation*}


\noindent
Since $b>0$, multiplying $a^n < b^n$ by $b$, we get:
\begin{equation}
  ba^n < b^{n+1} \label{2.1.23.1}
\end{equation}

\noindent
Since $a<b$, then $b-a>0$; and since $a>0$, then $a^n>0$.

\noindent
Also, since $(b-a)>0$ and $a^n>0$, we have: 
\begin{equation}
  (b-a)a^n > 0 \Rightarrow ba^n - a^{n+1} > 0 \Rightarrow ba^n > a^{n+1} \label{2.1.23.2}
\end{equation}

\noindent
Combining (\ref{2.1.23.1}) and (\ref{2.1.23.2}) together, we get:
\begin{equation*}
  a^{n+1} < ba^n < b^{n+1}
\end{equation*}

\noindent
Therefore, $a<b \Rightarrow a^{n+1} < b^{n+1}$, thus closing the induction.
\newline

\noindent
\textbf{\underline{Reverse direction:}} For $a>0$, $b>0$, and $n \in N$: If $a^n<b^n$, then $a<b$.
\newline

\noindent
We use induction.
\newline

\noindent
\textbf{\underline{Base case ($n=1$):}} This case is trivially true because it is given by the hypothesis: $a^1<b^1 \iff a < b$.
\newline

\noindent
\textbf{\underline{Inductive step ($n>1$):}} By the induction hypothesis, we have: 
\begin{equation}
  a^n<b^n \Rightarrow a < b \label{2.1.23.3}
\end{equation}

\noindent
The contrapositive of (\ref{2.1.23.3}) is:
\begin{equation}
  a \geq b \Rightarrow a^n \geq b^n \label{2.1.23.4}
\end{equation}

\noindent
Since $b>0$, multiplying $a^n \geq b^n$ by $b$, we get:
\begin{equation}
  ba^n \geq b^{n+1} \label{2.1.23.5}
\end{equation}

\noindent
Since $a \geq b$, then $a-b \geq 0$; and since $a>0$, then $a^n>0$.

\noindent
Also, since $(a-b) \geq 0$ and $a^n>0$, we have: 
\begin{equation}
  (a-b)a^n \geq 0 \Rightarrow a^{n+1} - ba^n \geq 0 \Rightarrow a^{n+1} \geq ba^n \label{2.1.23.6}
\end{equation}

\noindent
Combining (\ref{2.1.23.5}) and (\ref{2.1.23.6}) together, we get:
\begin{equation}
  a^{n+1} \geq ba^n \geq b^{n+1} \label{2.1.23.7}
\end{equation}

\noindent
Putting (\ref{2.1.23.4}), (\ref{2.1.23.5}), (\ref{2.1.23.6}), and (\ref{2.1.23.7}) together, we get:
\begin{equation}
  a \geq b \Rightarrow a^{n+1} \geq b^{n+1}  \label{2.1.23.8}
\end{equation}

\noindent
Taking the contrapositive of (\ref{2.1.23.8}), we get:
\begin{equation}
  a^{n+1} < b^{n+1} \Rightarrow a < b
\end{equation}

\noindent
This closes the induction and completes the proof.
\qed
\newline

\section*{2.2.16}

$V_{\epsilon}(a) = \{x \in R: |x-a| < \epsilon\}$

\noindent
$V_{\delta}(a) = \{x \in R: |x-a| < \delta\}$
\newline

\noindent
(i) $V_{\epsilon}(a) \cup V_{\delta}(a) = \{x \in R: |x-a| < \epsilon \ \text{and} \ |x-a| < \delta\}$
\newline

\noindent
Let $\gamma = \min(\epsilon, \delta)$.
\newline

\noindent
\textbf{\underline{Lower Bound:}}
$|x-a| < \epsilon \ \text{and} \ |x-a| < \delta \Rightarrow x > a-\epsilon \ \text{and} \ x > a-\delta \Rightarrow x > a-\gamma$
\newline

\noindent
\textbf{\underline{Upper Bound:}}
$|x-a| < \epsilon \ \text{and} \ |x-a| < \delta \Rightarrow x < a+\epsilon \ \text{and} \ x < a+\delta \Rightarrow x < a+\gamma$
\newline

\noindent
Therefore, $V_{\epsilon}(a) \cup V_{\delta}(a)$ is in the $\gamma$-neighbourhood of a.
\newline

\noindent
(ii) $V_{\epsilon}(a) \cap V_{\delta}(a) = \{x \in R: |x-a| < \epsilon \ \text{or} \ |x-a| < \delta\}$
\newline

\noindent
Let $\gamma = \max(\epsilon, \delta)$.
\newline

\noindent
\textbf{\underline{Lower Bound:}}
$|x-a| < \epsilon \ \text{or} \ |x-a| < \delta \Rightarrow x > a-\epsilon \ \text{or} \ x > a-\delta \Rightarrow x > a-\gamma$
\newline

\noindent
\textbf{\underline{Upper Bound:}}
$|x-a| < \epsilon \ \text{and} \ |x-a| < \delta \Rightarrow x < a+\epsilon \ \text{and} \ x < a+\delta \Rightarrow x < a+\gamma$
\newline

\noindent
Therefore, $V_{\epsilon}(a) \cap V_{\delta}(a)$ is in the $\gamma$-neighbourhood of a.
\newline

\section*{2.2.17}
Without loss of generality, assume that $b>a$. Let $\epsilon=\frac{b-a}{2}$
\newline

\noindent
Then, $U_{\epsilon}(a)=\{x \in R: |x-a| < \epsilon\}$, and $V_{\epsilon}(b)=\{x \in R: |x-b| < \epsilon\}$
\newline

\noindent
$U_{\epsilon}(a) \cap V_{\epsilon}(b) = \{x \in R: |x-a| < \epsilon \ \text{and} \ |x-b| < \epsilon\}$
\newline

\noindent
\textbf{\underline{Lower Bound:}}
$ $

$|x-a| < \epsilon \ \text{and} \ |x-b| < \epsilon$ 

$\implies x > a-\epsilon \ \text{and} \ x > b-\epsilon$ 

$\implies x > a-\frac{b-a}{2} \ \text{and} \  x > b - \frac{b-a}{2}$

$\implies x > \frac{3a-b}{2} \ \text{and} \  x > \frac{b+a}{2}$

$\implies x > \frac{b+a}{2}$
\newline

\noindent
\textbf{\underline{Upper Bound:}}
$ $

$|x-a| < \epsilon \ \text{and} \ |x-b| < \epsilon$ 

$\implies x < a+\epsilon \ \text{and} \ x < b+\epsilon$ 

$\implies x < a+\frac{b-a}{2} \ \text{and} \  x < b + \frac{b-a}{2}$

$\implies x < \frac{a+b}{2} \ \text{and} \  x < b+ \frac{b-a}{2}$

$\implies x < \frac{a+b}{2}$
\newline

\noindent
Since the lower bound and the upper bound do not intersect, then $U_{\epsilon}(a) \cap V_{\epsilon}(b) = \emptyset$.
\qed

\section*{2.3.4}
$S_4 = \{1-(-1)^n/n: n \in N\}$
\newline

\noindent
Let $S^\prime= \{-(-1)^n/n: n \in N\}$. Then: $S_4 = 1 + S^\prime $.
\newline

\noindent
$inf(S^\prime) = -(-1)^2/2 = -1/2$
\newline

\noindent
$sup(S^\prime) = -(-1)^1/1 = 1$
\newline

\noindent
\textbf{\underline{$inf(S_4)$:}}
\begin{equation*}
  inf(S_4) = 1 + inf(S^\prime) = 1 + (-1/2) = 1/2
\end{equation*}
\newline

\noindent
\textbf{\underline{$sup(S_4)$:}} 
\begin{equation*}
  sup(S_4) = 1 + sup(S^\prime) = 1 + 1 = 2
\end{equation*}

\section*{2.3.11}
\begin{proof}
  By the definition of the infimum and supremum we have $inf(S_0) \leq sup(S_0)$. So we need only show that (i)$inf(S) \leq inf(S_0)$, and (ii)$sup(S_0) \leq sup(S)$.
  \newline

  \noindent
  \textbf{\underline{i. $inf(S) \leq inf(S_0)$:}}
  We prove this by contradiction.
  
  \noindent
  Suppose that $inf(S_0) < inf(S)$. 
  
  \noindent
  Since $inf(S_0)$ is a infimum for $S_0$, then $\forall \epsilon > 0 \ \exists s \in S_0: s < inf(S_0) + \epsilon$.
  
  \noindent
  Taking $\epsilon = [inf(S) - inf(S_0)] / 2$
  
  \noindent 
  $\Rightarrow$ $\exists s \in S_0$ and $s < inf(S)$
  
  \noindent
  $\Rightarrow s \in S_0 \ and \  s \notin S$

  \noindent
  $\Rightarrow S_0 \not\subset S$.
  
  \noindent
  This is a contradiction. Therefore, we must conclude that $inf(S) \leq inf(S_0)$.
  \newline

  \noindent
  \textbf{\underline{ii. $sup(S_0) \leq sup(S)$:}}
  We prove this by contradiction.
  
  \noindent
  Suppose that $sup(S_0) > sup(S)$. 
  
  \noindent
  Since $sup(S_0)$ is a supremum for $S_0$, then $\forall \epsilon > 0 \ \exists s \in S_0: s > sup(S_0) + \epsilon$.
  
  \noindent
  Taking $\epsilon = [sup(S_0) - sup(S)] / 2$
  
  \noindent 
  $\Rightarrow$ $\exists s \in S_0$ and $s > sup(S)$
  
  \noindent
  $\Rightarrow s \in S_0 \ and \  s \notin S$

  \noindent
  $\Rightarrow S_0 \not\subset S$.
  
  \noindent
  This is a contradiction. Therefore, we must conclude that $sup(S_0) \leq sup(S)$.

\end{proof}

\section*{3.1.1.b}

$x_n := (-1)^n/n = (-1, 1/2, -1/3, 1/4, -1/5, ...)$

\section*{3.1.5.d}
Required to show: $\lim (\frac{n^2 - 1}{2n^2 + 3}) = \frac{1}{2}$

\begin{proof}
  Given any $\epsilon > 0$, we need to find $k(\epsilon)$ such that for all $n \geq k$: $|\frac{n^2 - 1}{2n^2 + 3} - \frac{1}{2}| < \epsilon$:

   \begin{align*}
    |\frac{n^2 - 1}{2n^2 + 3} - \frac{1}{2}| &< \epsilon \\
    |\frac{2n^2 - 2 - 2n^2 - 3}{4n^2 + 6}| &< \epsilon \\
    |\frac{-5}{4n^2 + 6}| &< \epsilon \\
    \frac{5}{4n^2 + 6} &< \epsilon \\
    \frac{5}{4n^2 + 6} &\leq \frac{5}{n^2} \leq \frac{5}{n} < \epsilon \\
  \end{align*}

  \noindent
  Taking $k(\epsilon)=5/\epsilon$ satisfies the required conditions.

\end{proof}

\section*{3.1.7}

\subsection*{a}
Required to show: $\lim (\frac{1}{ln(n+1)}) = 0$
\begin{proof}
  Given any $\epsilon > 0$, we need to find $k(\epsilon)$ such that for all $n \geq k$: $|\frac{1}{ln(n+1)} - 0| < \epsilon$:
  \begin{align*}
    |\frac{1}{ln(n+1)} - 0| &< \epsilon \\
    \frac{1}{ln(n+1)} &< \epsilon \\
    ln(n+1) &> \frac{1}{\epsilon} \\
    n+1 &> e^\frac{1}{\epsilon} \\
    n &>  e^\frac{1}{\epsilon} - 1 \\
  \end{align*}

  Taking $k(\epsilon)= e^\frac{1}{\epsilon} - 1$ satisfies the required conditions.

\end{proof}

\subsection*{b}
i. $k(1/2) = e^2 - 1 = 7$

\noindent
ii. $k(1/10) = e^{10} - 1 = 22026$

\section*{3.1.9}
\begin{proof}
  $\lim(x_n)=0 \Rightarrow$ $\forall \epsilon > 0$ there exists $k(\epsilon)$ such that for all $n \geq k:\ |x_n - 0| < \epsilon$.
  
  \noindent
  $\Rightarrow$ $\forall \epsilon > 0$ there exists $k(\epsilon)$ such that for all $n \geq k: x_n < \epsilon$ (because $x_n > 0$).

  \noindent
  $\Rightarrow$ $\forall \epsilon > 0$ there exists $k(\epsilon)$ such that for all $n \geq k: \sqrt{x_n} - 0< \sqrt{\epsilon} = \epsilon^{'}$.

  \noindent
  Since $\epsilon^{'}$ can take on any value greater than zero, then by the definition of the limit of a sequence this shows that $\lim(\sqrt{x_n})=0$.
  \noindent 

\end{proof}


\section*{3.1.12}
Required to show: $\lim (\sqrt{n^2 + 1} - n) = 0$
\begin{proof}
  Given any $\epsilon > 0$, we need to find $k(\epsilon)$ such that for all $n \geq k$: $|(\sqrt{n^2 + 1} - n) - 0| < \epsilon$:
  \begin{align*}
    |(\sqrt{n^2 + 1} - n) - 0| &< \epsilon \\
    |\sqrt{n^2 + 1} - n| &< \epsilon \\
    \sqrt{n^2 + 1} - n \leq \sqrt{(n + 1/n)^2} - n &< \epsilon \\
    \sqrt{n^2 + 1} - n \leq (n + 1/n) - n &< \epsilon \\
    \sqrt{n^2 + 1} - n \leq 1/n &< \epsilon \\
  \end{align*}

  Taking $k(\epsilon)= \frac{1}{\epsilon}$ satisfies the required conditions.

\end{proof}

\section*{3.2.2}
\subsection*{a.}
% $X = (n)$, $Y = (-n) \implies X+Y = (0)$.
$X = (sin^2(n))$, $Y = (cos^2(n)) \implies X+Y = (1)$.

\subsection*{b.}
$X = ((-1)^n)$, $Y = ((-1)^{n+1}) \implies X*Y = (-1)$.

\section*{3.2.13}

Multiply and divide by the complex conjugate to put the sequence into a more favorable form:

\begin{flalign*}
  (\sqrt{(n+a)(n+b)}-n) * \frac{\sqrt{(n+a)(n+b)}+n}{\sqrt{(n+a)(n+b)}+n} &= \frac{(n+a)(n+b) - n^2}{\sqrt{(n+a)(n+b)}+n} &&\\ 
  &= \frac{n^2 + an + bn - ab - n^2}{\sqrt{(n+a)(n+b)}+n} &&\\ 
  &= \frac{an + bn - ab}{\sqrt{(n+a)(n+b)}+n} &&\\ 
  &= \frac{an + bn - ab}{\sqrt{(n+a)(n+b)}+n} &&\\ 
  &= \frac{a + b - \frac{ab}{n}}{\sqrt{(1+\frac{a}{n})(1+\frac{b}{n})}+1} &&\\ 
\end{flalign*}

\noindent
Now taking the limit:

\begin{flalign*}
  \lim{\frac{a + b - \frac{ab}{n}}{\sqrt{(1+\frac{a}{n})(1+\frac{b}{n})}+1}} &= \frac{\lim{a} + \lim{b} - \lim{\frac{ab}{n}}}{\lim{\sqrt{1+\frac{a}{n}}}*\lim{\sqrt{1+\frac{b}{n}}} +\lim{1}} &&\\
  &= \frac{a+b - 0}{1*1+1} = \frac{a+b}{2}
\end{flalign*}


\section*{3.2.14}
\subsection*{a.}

We have:

$0 \leq 1/n^2 \leq 1/n$

\noindent
$\implies n^0 \leq n^{1/n^2} \leq n^{1/n}$

\noindent
And we know the following limits:

$\lim(n^0) = 1 \ \land \ \lim{(n^{1/n)} = \lim{(n^{\lim{1/n}}})} = \lim{(n^0)}=1$

\noindent
$\implies \lim{(n^{1/n^2})} = 1$ (By the squeeze theorem).

\subsection*{b.}

We have:

$1 \leq n! \leq n^n $

\noindent
$\implies 1^{1/n^2} \leq (n!)^{1/n^2} \leq (n^n)^{1/n^2}$

\noindent
And we know the following limits:

$\lim{(1^{1/n^2})} = 1^0 = 1 \ \land \ \lim{((n^n)^{1/n^2})} = \lim{(n^{1/n})} = n^0 = 1$

\noindent
$\implies \lim{((n!)^{1/n^2})} = 1$ (By the squeeze theorem).

\section*{3.2.22}
The definition given for $(y_n)$ is exactly the definition of the limit.

\noindent
Therefore, $\lim{(y_n)} = (x_n)$.

\noindent
And, because $(x_n)$ is convergent, then $(y_n)$ must also be convergent.


\section*{3.3.3}

We first show that $(x_n)$ is bounded below by 2 using induction.

\begin{proof}
  $ $

\textbf{\underline{Base case ($n=1$):}} This case is trivially true since $x_1 \geq 2$.
\newline

\textbf{\underline{Inductive step ($n>1$):}} Assume $x_n \geq 2$

$\implies x_{n+1} = 1 + \sqrt{x_n - 1} \geq 1 + \sqrt{2 - 1} = 2$.

\end{proof}


\noindent
Next, we show that $(x_n)$ is decreasing using induction.

\begin{proof}
  $ $

\textbf{\underline{Base case ($n=1$):}} $x_2 = 1 + \sqrt{x_1-1} \leq x_1 f$ for $x \geq 2$.
\newline

\textbf{\underline{Inductive step ($n>1$):}} Assume $x_{n+1} \leq x_n$

$\implies x_{n+2} = 1 + \sqrt{x_{n+1} - 1} \leq 1 + \sqrt{x_n - 1} = x_{n+1}$.

\end{proof}


\noindent
Finally, to find the limit we note that at the limit we have:

$x_{n+1} = x_n$

\noindent
$\implies lim{(x_n) = x = 1 + \sqrt{x-1}}$

\noindent
$\implies x = 2$.


\section*{3.4.9}

\begin{proof}{(By contradiction)}
  $ $

  Suppose $\lim{X} = a \neq 0$.

  \noindent
  $\implies$ All \textbf{subsequences} of $X$ must converge to a (By theorem 3.4.2).

  \noindent
  $\implies$ All \textbf{subsequences of subsequences} of $X$ must converge to a (Again by theorem 3.4.2).
  \newline
  
  But this is a contradiction to our initial assumption that all subsequences of $X$ contain a subsequence that converges to zero. Therefore, we conclude that $lim{X} = 0$.

\end{proof}


\section*{3.5.9}
\begin{proof}
  $ $

  \noindent
  \begin{flalign*}
    m>n \implies |x_{m} - x_{n}| &< r^n + r^{n+1} + \ldots + r^{m-1} &&\\ 
    & \leq \frac{r^n}{1-r} &&\\
    & < \epsilon \ \ \forall \epsilon > 0 \ and \ n \geq H(\epsilon) \ \text{(Because} \lim{\frac{r^n}{1-r}} = 0 \text{)}.
  \end{flalign*}

  \noindent
  $\implies (x_n)$ is a Cauchy sequence.

\end{proof}


\section*{3.7.9}
\subsection*{a.}
\begin{proof}
  $ $

  The sequence (cos(n)) does not converge to zero.
  
  \noindent
  $\implies \sum_{n=1}^{\infty} cos(n)$ is divergent.

\end{proof}

\subsection*{b.}
\begin{proof}
  $ $

  Let $X := (\frac{1}{n^2})$ and $Y = (\frac{cos(n)}{n^2})$. 
  
  \noindent
  $\implies X$ is convergent and $x_n < y_n$

  \noindent
  $\implies Y$ is convergent (By the comparison test).
  
\end{proof}


\section*{4.2.14}
\begin{proof}
  $ $

  $\lim_{x\to c}{f} = L \implies |f(x) - L| <   \epsilon$
  \newline

  And we have:
  $|f(x) - L| \geq ||f(x)| - |L||$
  \newline

  Therefore: 
  $||f(x)| - |L|| < \epsilon$
  \newline

  This implies that $\lim_{x\to c}{|f|} = |L|$

\end{proof}


\section*{5.4.2}

\textbf{\underline{On A = [1, $\infty$):}}
\begin{proof}
  $ $
  
  Let $x, u \geq 1$.
  \begin{flalign*}
    \implies |f(x) - f(u)| &= |\frac{1}{x^2} - \frac{1}{u^2}| &&\\
    &= |\frac{u^2 - x^2}{u^2x^2}| &&\\
    &= |\frac{(u+x)(u-x)}{u^2x^2}| &&\\
    &= |\frac{(u+x)(u-x)}{u^2x^2}| &&\\
    &= \frac{(u+x)}{u^2x^2}|u-x| &&\\
    &= (\frac{1}{ux^2} + \frac{1}{u^2x})|u-x| &&\\
    &\leq 2|u-x| &&\\
  \end{flalign*}

  \noindent
  $\implies \forall \epsilon > 0 \ \exists \delta(\epsilon) = \frac{\epsilon}{2}: (\forall x,u \in A: |x-u|< \delta \implies |f(x) - f(u)| < \epsilon)$

  \noindent
  $\implies f$ is uniformly continuous on $A$.

\end{proof}

\noindent
\textbf{\underline{On B = [0, $\infty$):}}
\begin{proof}
  $ $

  Take $x=\frac{1}{n}$ and $u=\frac{1}{n+1}$.

  \noindent
  $\implies |x - u| = |\frac{1}{n} - \frac{1}{n+1}| = |\frac{1}{n^2 + n}|$

  \noindent
  $\implies lim(|x-u|) = 0$.
  \newline

  But we also have:

  $|f(x)-f(u)| = |n^2 - (n+1)^2| = |n^2 - n^2 -2n - 1 | = 2n + 1 \geq 1$.
  \newline
  
  \noindent
  This shows that if we take $\epsilon$ to be any value $< 1$, then there is no corresponding $\delta$ that can satisfy the condition $|x-u|<\delta \implies |f(x) - f(u)|$.

  \noindent
  $\implies f$ is not uniformly continuous on $B$.

\end{proof}

\section*{6.2.14}
\begin{proof}{(By Contradiction)}
  $ $

  Suppose there exists $a,b \in I$ such that $f^\prime(a) > 0$ and $f^\prime(b) < 0$.

  \noindent
  $\implies \exists c \in I: f^\prime(c) = 0$ \qquad (By Darboux's theorem)
  \newline

  \noindent
  This is a contradiction. Therefore, we must conclude that either $f^\prime(x) > 0 \ \forall x \in I$ or $f^\prime(x) < 0 \ \forall x \in I$.
  
\end{proof}


\section*{6.2.15}
\begin{proof}
  $ $

  Let $f^\prime(x) \leq C$, and $x_1, x_2 \in I$.
  
  \begin{flalign*}
    \implies \exists z \in I: |f(x_1) - f(x_2)| &= |f^\prime(z)(x_1-x_2)| \qquad \text{(By the Mean Value Theorem)} &&\\
    &\leq C|x_1-x_2| &&\\
  \end{flalign*}

  \noindent
  $\implies f$ satisfies the Lipschitz condition.

\end{proof}



\section*{7.2.18}
\begin{proof}
  $ $

  Let $M := sup (f)$ and $p \in [a,b]$ such that $f(p)=M$.
  \newline

  Since $f$ is continuous at $p$, then given an $\epsilon > 0$, there exists $\delta > 0$ such that:
  \begin{align*}
    |x-p| \leq \delta &\implies |f(x)-f(p)| \leq \epsilon &&\\
    &\implies f(p) - \epsilon \leq f(x) \leq f(p) &&\\
    &\implies M - \epsilon \leq f(x) \leq M &&\\
    &\implies (M - \epsilon)^n \leq f(x)^n \leq M^n &&\\
    &\implies 2\delta (M - \epsilon)^n \leq \int_{p-\delta}^{p+\delta} f(x)^n \leq (b-a) M^n &&\\
    &\implies 2\delta (M - \epsilon)^n \leq \int_{p-\delta}^{p+\delta} f(x)^n \leq \int_{a}^{b} f(x)^n \leq (b-a) M^n &&\\
    &\implies 2\delta (M - \epsilon)^n \leq \int_{a}^{b} f(x)^n \leq (b-a) M^n &&\\
    &\implies (2\delta)^{\frac{1}{n}} (M - \epsilon) \leq \left( \int_{a}^{b} f(x)^n \right)^{\frac{1}{n}} \leq (b-a)^{\frac{1}{n}} M &&\\
    &\implies (M - \epsilon) \leq M_n \leq  M.
  \end{align*}

  \noindent
  $\implies \lim{M_n} = M$.

\end{proof}

\section*{7.4.7}
\subsection*{a.}
\begin{proof}
  $ $

  Let $P = [0, \frac{1}{2}-\epsilon, \frac{1}{2}+\epsilon, 1]$

  \noindent
  \begin{flalign*}
    \implies &L(g;P) = 0*(\frac{1}{2}-\epsilon) + 0*(2\epsilon) + 1*(\frac{1}{2}-\epsilon) = \frac{1}{2} - \epsilon &&\\ 
    &U(g;P) = 0*(\frac{1}{2}-\epsilon) + 1*(2\epsilon) + 1*(\frac{1}{2}-\epsilon) = \frac{1}{2} + \epsilon &&\\ 
  \end{flalign*}

  \noindent
  $\implies U(g;P) - L(g;P) = 2\epsilon$


  \noindent
  $\implies$ The Darboux integral of $g$ on $[0,1]$ is $\frac{1}{2}$.
  
\end{proof}

\subsection*{b.}

Let $P = [0, \frac{1}{2}-\epsilon, \frac{1}{2}+\epsilon, 1]$

\noindent
\begin{flalign*}
  \implies &L(g;P) = 0*(\frac{1}{2}-\epsilon) + 0*(2\epsilon) + 1*(\frac{1}{2}-\epsilon) = \frac{1}{2} - \epsilon &&\\ 
  &U(g;P) = 0*(\frac{1}{2}-\epsilon) + 13*(2\epsilon) + 1*(\frac{1}{2}-\epsilon) = \frac{1}{2} + 25\epsilon &&\\ 
\end{flalign*}

\noindent
$\implies U(g;P) - L(g;P) = 26\epsilon$

\noindent
$\implies$ $g$ is Darboux integrable on $[0,1]$ with an integral value of $\frac{1}{2}$.



\section*{8.1.23}
\begin{proof}
  $ $

  Let $M$ be the larger of the upper bounds on $f_n$ and $g_n$.
  
  \noindent
  \begin{flalign*}
    \implies |f_ng_n - fg| &= |f_ng_n + f_ng - f_ng - fg| &&\\
    &\leq  |f_ng_n - f_ng| + |f_ng - fg| &&\\
    &=  |f_n||g_n - g| + |g||f_n - f| &&\\
    &\leq  M|g_n - g| + M|f_n - f| &&\\
    &< \epsilon. &&\\
  \end{flalign*}

  \noindent
  $\implies lim{(f_ng_n)} = fg$.

\end{proof}






% \newline
\noindent\rule{\textwidth}{1pt}
\section*{Extras}

\begin{customthm}{3.1.4}[Uniqueness of Limits]
  $ $

  A sequence in $\mathbb{R}$ can have at most one limit.
\end{customthm}

\begin{proof}{(By contradiction)}
  $ $

  Suppose for the sake of contradiction that $x^\prime$ and $x^{\prime\prime}$ are two limits of the sequence $(x_n)$ and $x^\prime \neq x^{\prime\prime}$.

  \noindent
  $\implies 
  \forall \epsilon > 0: \ 
  (\exists K^\prime: \forall n \geq K^\prime: |x_n - x^\prime| < \epsilon)
  \land \ 
  (\exists K^{\prime\prime}: \forall n \geq K^{\prime\prime}: |x_n - x^{\prime\prime}| < \epsilon)$
  
  \noindent
  \begin{flalign*}
    \implies \forall n \geq K = max(K^\prime, K^{\prime\prime}): |x^{\prime} - x^{\prime\prime}| &= |x^{\prime} - x_n + x_n - x^{\prime\prime}| &&\\
    &\leq |x^{\prime} - x_n| + |x_n - x^{\prime\prime}| &&\\
    & < \epsilon + \epsilon = 2\epsilon.
  \end{flalign*}

  \noindent
  $\implies x^\prime = x^{\prime\prime}$, since we can make $\epsilon$ as small as we wish.
  \newline

  \noindent
  But this is a contradiction to our initial assumption that $x^\prime \neq x^{\prime\prime}$. Therefore, we conclude that a sequence in $\mathbb{R}$ can have at most one limit.

\end{proof}

% 

\begin{customthm}{3.2.2}
  $ $

  A convergent sequence of real numbers is bounded.
\end{customthm}

\begin{proof}
  $ $

  Let $\lim{(x_n)} = x$ and $\epsilon := 1$.

  \noindent
  $\implies \exists K=K(1): \forall n \geq K: |x_n - x| < \epsilon = 1$

  \noindent
  \begin{flalign*}
    \implies |x_n| &= |x_n - x + x| &&\\
    & \leq |x_n - x| + |x| &&\\
    & < 1 + |x|. &&\\
  \end{flalign*}
  
  Define $M := sup\{ |x_1|, |x_2|, \dots, |x_{K-1}|, 1 + |x| \}$.
  
  \noindent
  $\implies \forall n \in \textbf{N}: |x_n| \leq M$.

  \noindent
  $\implies$ The sequence $(x_n)$ is bounded. 

\end{proof}

% 

\begin{customthm}{3.4.8}[Bolzano-Weierstrass Theorem]
  $ $

  A bounded sequence of real numbers has a convergent subsequence.
\end{customthm}

\begin{proof}
  $ $

  Let $X$ be our bounded sequence.
  
  By the Monotone Subsequence Theorem, $X$ has a subsequence $X^\prime$ that is monotone.

  Since $X$ is bounded, then so is $X^\prime$.

  Since $X^\prime$ is monotone and bounded, then, by the Monotone Convergence Theorem, it is convergent.
  
\end{proof}

% 

\begin{customthm}{5.3.7}[Bolzano’s Intermediate Value Theorem]
  $ $

  Let $I$ be an interval and let $f: I \rightarrow \textbf{R}$ be continuous on $I$. If $a, b \in I$ and if $k \in \textbf{R}$ satisfies $f(a) < k < f(b)$, then there exists a point $c \in I$ between $a$ and $b$ such that $f(c) = k$.
\end{customthm}

\begin{proof}{(By Cases)}
  $ $

  \underline{Case 1: $a < b$:}

  Define $g(x):=f(x) - k$. 

  $\implies g(a) < 0 < g(b)$

  $\implies \exists c \ \text{where} \ a < c < b:  g(c) = 0$ (By the Location of Roots Theorem)

  $\implies \exists c: f(c) = g(c) + k = k$.
  \newline


  \underline{Case 2: $b < a$:}
  
  Define $h(x):=k-f(x)$.

  $\implies h(b) < 0 < h(a)$

  $\implies \exists c \ \text{where} \  b < c < a:  g(c) = 0$ (By the Location of Roots Theorem)

  $\implies \exists c: f(c) = k - h(c) = k$.
  
\end{proof}


% 

\begin{customthm}{6.2.12}[Darboux's Theorem]
  $ $

  If f is differentiable on $I = [a, b]$ and if $k$ is a number between $f^{\prime}(a)$ and $f^{\prime}(b)$, then there is at least one point $c$ in $(a, b)$ such that $f^{\prime}(c) = k$.
\end{customthm}

\begin{proof}{}
  $ $

  Suppose that $f^\prime(a) < k < f^\prime(b)$. 
  
  Define $g(x):=kx-f(x)$ for $x \in I$.

  Since $g$ is continuous, it attains a maximum value on $I$.

  Since $g^\prime(a) = k - f^\prime(a) > 0$, then the maximum of $g$ does not occur at $x=a$.

  Similarly, since $g^\prime(b) = k - f^\prime(b) < 0$, then the maximum of $g$ does not occur at $x=b$.

  Therefore, by the Interior Extremum theorem, $g$ attains its maximum at some point $c$ in $(a,b)$ where $g^\prime(c) = k - f^\prime(c) = 0$. Hence $f^\prime(c) = k$.
  
  
\end{proof}

\end{document}
